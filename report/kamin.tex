\documentstyle[cprog]{report}
\includeonly{future}
\begin{document}
\setlength{\textwidth}{7.4in}
\setlength{\oddsidemargin}{1in}
\setlength{\evensidemargin}{1in}
\setlength{\topmargin}{0in}
\title{The Kamin Interpreters in C++}
\author{Tim Budd}
\maketitle
\begin{abstract}
This paper describes a series of interpreters for the languages used in the 
book ``Programming Languages: An Interpreter-Based Approach'' by Samuel 
Kamin (Addison-Wesley, 1989).  Unlike the interpreters provided by Kamin,
which are written in Pascal, 
these interpreters are written in C++.
It is my belief that the use of inheritance in C++ better
illustrates the unique features of each of the several languages.
In the Pascal versions of the interpreters the differences between the
various interpreters, although small, are scattered throughout the code.
In the C++ versions differences are produced using only the mechanism of
subclassing.  This means that the vast majority of code remains the same, 
and differences can be much more precisely isolated.

The chapters in this report correspond to the chapters in the original text.
Where motivational or background material is provided in that source it is
generally omitted here.  A major exception is in those places (chiefly
chapters 3, 7 and 8) where I have selected a syntax slightly different 
from that provided by Kamin.

The use of an Object-Oriented language for the interpreters may seem a bit
incongruous, since Object-Oriented programming is not discussed until
Chapter 7.  Nevertheless, I think the benefits of programming the
interpreters in C++ outweighs this problem.
\end{abstract}

\chapter{The Basic Interpreter}

The structure of our basic interpreter\footnote{It should be noted that our
basic interpreter is not an interpreter for Basic.}
differs somewhat from that described
by Kamin.  Our interpreter is structured around a small main program which
manipulates three distinct types of data structures.  The main program is
shown in Figure~\ref{main}, and will be discussed in more detail in the
next section.  
Each of the three main data structures is represented by a C++ class, such
is subclassed in various ways by the different interpreters.
The three varieties of data structures are the following:
\begin{itemize}
\item
{\bf Readers}.  
Instances of this class prompt the user for input values,
and break the input into a structure of unevaluated components.
A single instance of either the class {\sf Reader} or a subclass
is created during the initialization
process for each interpreter.  The base reader class is subclassed in those
interpreters which introduce new syntactic elements (such as quoted lists
in Lisp or vectors in APL).
\item
{\bf Environments}.
An Environment is a data structure used to maintain a collection of
symbol-value pairs, such as the global run-time environment or the values
of arguments passed to a function.  Values can be added to an environment,
and the existing binding of a symbol to a value can be changed to a new
value.
\item
{\bf Expressions}.
Expressions represent the heart of the system, and the differences between
the various interpreters is largely found in the various different types of
expressions they manipulate.  Expressions know how to ``evaluate
themselves'' where the meaning of that expression is determined by each
type of expression.  In addition expressions also know how to print their
value, and that, too, differs for each type of expression.
\end{itemize}

In subsequent sections we will explore in more detail each of these data
structures.

\section{The Main Program}

Figure~\ref{main} shows the main program,\footnote{I have omitted the
``include'' directives and certain global declarations from this figure.
The complete code can be found elsewhere (where?).}
which defines the top level
control for the interpreters.  The same main program is used for each of
the interpreters.  Indeed, the vast majority of code remains constant
throughout the interpreters.

\begin{figure}
\begin{cprog}
main() {
	Expr entered;	// expression as entered by users

	// common initialization
	emptyList = new ListNode(0, 0);
	globalEnvironment = new Environment(emptyList, emptyList, 0);
	valueOps = new Environment(emptyList, emptyList, 0);
	commands = new Environment(emptyList, emptyList, valueOps);

	// language-specific initialization
	initialize();

	// the read-eval-print loop
	while (1) {
		entered = reader->promptAndRead();

		// see if expression is quit
		Symbol * sym = entered()->isSymbol();
		if (sym && (*sym == "quit"))
			break;

		// nothing else, must just be an expression
		entered.evalAndPrint(commands, globalEnvironment);
		}
}
\end{cprog}
\caption{The Read-Eval-Print Loop for the interpreters}\label{main}
\end{figure}

The structure of the main program is very simple.
To begin, a certain amount of initialization is necessary.  There are four
global variables found in all the interpreters.  The variable
{\sf emptyList} contains a list with no elements.  (We will return to a
discussion of lists in Section~\ref{listsec}).
The three environments {\sf globalEnvironment}, {\sf valueOps} and {\sf
commands} represent the top-level context for the interpreters.
The {\sf globalEnvironment} contains those symbols that are accessible at
the top level.  The {\sf valueOps} are those operations that can be
performed at any level, but which are not symbols themselves that can be
manipulated by the user.  Finally {\sf commands} are those functions that
can be invoked only at the top level of execution.  That is, commands
cannot be executed within function definitions.

Following the common initialization the function {\sf initialize} is called
to provide interpreter-specific initialization.  This chiefly consists of
adding values to the three environments.  This function is changed 
in each of the various interpreters.

The heart of the system is a single loop, which executes until the user
types the directive {\sf quit}.\footnote{The reader data structure will
trap end-of-input signals, and if detected acts as if the user had typed
the {\sf quit} directive.}  The reader (which must be defined as part of the
interpreter-specific initialization) requests a value from the user.
After testing for the the {\sf quit} directive, the entered expression is
evaluated.  We will defer an explaination of the {\sf evalAndPrint} method 
until Section~\ref{exprsec}, merely noting here that 
it evaluates the expression the user has entered and prints the result.
The read-eval-print cycle then continues.

\section{Readers}

Readers are implemented by instances of class {\sf Reader}, shown in
Figure~\ref{readerclass}.  The only public function performed by this class
is provided by the method {\sf promptAndRead}, which prints the interpreter
prompt, waits for input from the user, and then parses the input into a
legal, but unevaluated, expression (usually a symbol, integer or 
list-expression).  These actions are implemented by a variety of utility
routines, which are declared as {\sf protected} so that they may be
made available to later subclasses.

\begin{figure}
\begin{cprog}
class Reader {
public:
	Expression * promptAndRead();

protected:
	char buffer[1000];	// the input buffer
	char * p;		// current location in buffer

	// general functions
	void printPrimaryPrompt();
	void printSecondaryPrompt();
	void fillInputBuffer();
	int  isSeparator(int);
	void skipSpaces();
	void skipNewlines();

	// functions that can be overridden
	virtual Expression * readExpression();

	// specific types of expressions
	int readInteger();
	Symbol * readSymbol();
	ListNode * readList();
};
\end{cprog}
\caption{Class Description of the reader class}\label{readerclass}
\end{figure}

\begin{figure}
\begin{cprog}
Expression * Reader::promptAndRead()
{
	// loop until the user types something
	do {
		printPrimaryPrompt();
		fillInputBuffer();
		} while (! *p);

	// now that we have something, break it apart
	Expression * val = readExpression();

	// make sure we are at and of line
	skipSpaces();
	if (*p) {
		error("unexpected characters at end of line:", p);
		}

	// return the expression
	return val;
}
\end{cprog}
\caption{The Method {\sf promptandRead} from class {\sf Reader}}\label{promptAndRead}
\end{figure}

The code that implements this data structure is relatively
straight-forward, and most of it will not be presented here.
The main method is the
single public-accessible routine {\sf promptAndRead}, which is shown in
Figure~\ref{promptAndRead}.  
This method loops until the user enters an expression.  The method
{\sf fillInputBuffer} places the instance pointer {\sf p} at the first 
non-space character (also stripping out comments).   Thus lines containing 
only spaces,
newlines, or comments are handled quickly here, and cause no further
action.  Also, as have noted previously, an end-of-input indication is
caught by the method {\sf fillInputBuffer}, which then places the 
{\sf quit} command in the input buffer.
The method {\sf readExpression} (Figure~\ref{readExpression}) is the parser 
used to break the input into
an unevaluated expression.  This method is declared {\sf virtual}, and thus
can be redefined in subclasses.  The base method recognizes only integers,
symbols, and lists.   The routine to read a list recursively calls the
method to read an expression.

\begin{figure}
\begin{cprog}
Expression * Reader::readExpression()
{
	// see if it's an integer
	if (isdigit(*p))
		return new IntegerExpression(readInteger());

	// might be a signed integer
	if ((*p == '-') && isdigit(*(p+1))) {
		p++;
		return new IntegerExpression(- readInteger());
		}

	// or it might be a list
	if (*p == '(') {
		p++;
		return readList();
		}
	
	// otherwise it must be a symbol
	return readSymbol();
}

ListNode * Reader::readList()
{
	// skipNewlines will issue secondary prompt
	// until a valid character is typed
	skipNewlines();

	// if end of list, return empty list
	if (*p == ')') {
		p++;
		return emptyList;
		}

	// now we have a non-empty character
	Expression * val = readExpression();
	return new ListNode(val, readList());
}
\end{cprog}
\caption{The method {\sf readExpression} and {\sf readList}}\label{readExpression}
\end{figure}

\section{Environments}

As we have noted already, the {\sf Environment} data structure is used to
maintain symbol-value pairings.  In addition to the global environments
defined during initialization, environments are created for argument lists
passed to functions, and in various other contexts by some of the later
interpreters.  Environments can be linked together, so that if a symbol is
not found in one environment another can be automatically searched.
This facilitates lexical scoping, for example.

For reasons having to do with memory management, the {\sf Environment} data
structure, shown in Figure~\ref{environment}, is declared as a subclass of
the class {\sf Expression}.  Unlike other expressions, however,
environments are never directly manipulated by the user.
Also for memory management reasons, there is a class {\sf Env} declared
which can maintain a pointer to an environment.   The two methods defined
in class {\sf Env} set and return this value.   Anytime a pointer is to be
maintained for any period of time, such as the link field in an
environment, it is held in a variable declared as {\sf Env} rather than as
a pointer directly.
Finally the overridden virtual methods {\sf isEnvironment} and {\sf free} 
in class {\sf Environment} are also
related to memory management, and we will defer a discussion of these until
the next section.

\begin{figure}
\begin{cprog}
class Environment : public Expression {
private:
	List theNames;
	List theValues;
	Env theLink;

public:
	Environment(ListNode *, ListNode *, Environment *);

	// overridden methods
	virtual Environment * isEnvironment();
	virtual void free();

	// new methods
	Expression * lookup(Symbol *);
	void add(Symbol *, Expression *);
	void set(Symbol *, Expression *);
};

class Env : public Expr {
public:
	operator Environment * ();
	void operator = (Environment * r);
};
\end{cprog}
\caption{The {\sf Environment} data structure}\label{environment}
\end{figure}

The three methods used to manipulate environments are {\sf lookup}, {\sf
add} and {\sf set}.   The first attempts to find the value of the symbol
given as argument, returning a null pointer if no value exists.
The method {\sf add} adds a new symbol-value pair to the front of the current
environment.  The method {\sf set} is used to redefine an existing value.
If the symbol is not found in the current environment and there is a valid
link to another environment the linked environment is searched.  If the
link field is null (that is, there is no next environment), the symbol and
valued are {\sf add}ed to the current environment.

Environments are implemented using the List data structure, a form of 
Expression we will describe in more detail in Section~\ref{listsec}.
Two parallel lists contain the symbol keys and their associated values.
For the moment it is only necessary to characterize lists by four
operations.   A list is composed of list nodes (elements of class {\sf
ListNode}).  Each node contains an expression (the head) and, recursively,
another list.  The special value {\sf emptyList}, which we have already
encountered, terminates every list.
The operation {\sf head} returns the first element of a list node.  
When provided with an argument, the operation {\sf head} can be used to
modify this first element.
The operation {\sf tail} returns the remainder of the list.
Finally the operation {\sf isNil} returns true if and only if the list is
the empty list.

\begin{figure}
\begin{cprog}
Expression * Environment::lookup(Symbol * sym)
{
	ListNode * nameit = theNames;
	ListNode * valueit = theValues;

	while (! nameit->isNil()) {
		if (*sym == nameit->head())
			return valueit->head();
		nameit = nameit->tail();
		valueit = valueit->tail();
		}

	// otherwise see if we can find it on somebody elses list
	Environment * link = theLink;
	if (link) return link->lookup(sym);

	// not found, return nil value
	return 0;
}
\end{cprog}
\caption{The method {\sf lookup} in class {\sf Environment}}\label{lookup}
\end{figure}

Figure~\ref{lookup} shows the method {\sf lookup}, which is defined in
terms of these four operations.  The while loop cycles over the list of
keys until the end (empty list) is reached.  Each key is tested against the
argument key, using the equality test provided by the class {\sf Symbol}.
Once a match is found the associated value is returned.

If the entire list of names is searched with no match found,
if there is a link to another environment the lookup message is passed to
that environment.  If there is no link, a null value is returned.

The routine to add a new value to an environment (Figure~\ref{add}) merely 
attaches a new name and value to the beginning of the respective lists.  
Note by attaching to be beginning of a list this will hide any existing 
binding of the name, although such a situation will not often occur.  
The method {\sf set} searches for an existing binding, replacing it if
found, and only adding the new element to the final environment if no 
binding can be located.

\begin{figure}
\begin{cprog}
void Environment::add(Symbol * s, Expression * v)
{
	theNames = new ListNode(s, (ListNode *) theNames);
	theValues = new ListNode(v, (ListNode *) theValues);
}

void Environment::set(Symbol * sym, Expression * value)
{
	ListNode * nameit = theNames;
	ListNode * valueit = theValues;

	while (! nameit->isNil()) {
		if (*sym == nameit->head()) {
			valueit->head(value);
			return;
			}
		nameit = nameit->tail();
		valueit = valueit->tail();
		}

	// see if we can find it on somebody elses list
	Environment * link = theLink;
	if (link) {
		link->set(sym, value);
		return;
		}

	// not found and we're the end of the line, just add
	add(sym, value);
}
\end{cprog}
\caption{Methods used to Insert into an environment}\label{add}
\end{figure}

\section{Expressions}\label{exprsec}

The class {\sf Expression} is a root for a class hierarchy that contains 
the majority of classes defined in these interpreters.  Figure~\ref{classpic} 
shows a portion of this class hierarchy.  We have already seen that
environments are a form of expression, as are integers, symbols, lists and
functions. 

\setlength{\unitlength}{5mm}
\begin{figure}
\begin{picture}(16,10)(-4,-3)
\put(-3.5,0){\sf Expression}
\put(0,0.2){\line(1,0){1}}
\put(1,0){\sf Function}
\put(0,0.2){\line(1,1){1}}
\put(1,1){\sf List}
\put(0,0.2){\line(1,2){1}}
\put(1,2){\sf Symbol}
\put(0,0.2){\line(1,3){1}}
\put(1,3){\sf Integer}
\put(0,0.2){\line(1,-2){1}}
\put(1,-2){\sf Environment}
\put(4,0.2){\line(1,0){1}}
\put(5,0){\sf BinaryFunction}
\put(4,0.2){\line(1,1){1}}
\put(5,1){\sf UnaryFunction}
\put(4,0.2){\line(1,2){1}}
\put(5,2){\sf BeginStatement}
\put(4,0.2){\line(1,3){1}}
\put(5,3){\sf SetStatement}
\put(4,0.2){\line(1,4){1}}
\put(5,4){\sf WhileStatement}
\put(4,0.2){\line(1,5){1}}
\put(5,5){\sf IfStatement}
\put(4,0.2){\line(1,6){1}}
\put(5,6){\sf DefineStatement}
\put(4,0.2){\line(1,-1){1}}
\put(5,-1){\sf UserFunction}
\put(9.5,0.2){\line(1,0){1}}
\put(10.5,0){\sf IntegerBinaryFunction}
\end{picture}
\caption{The {\sf Expression} class Hierarchy in Chapter 1}\label{classpic}
\end{figure}

\subsection{The Abstract Class}

The major purposes of the abstract class {\sf Expression} 
(Figure~\ref{expression}) are to perform memory management functions, to
permit conversions from one type to another in a safe manner, and to define
protocol for evaluation and printing of expression values.  The latter is
easist to dismiss.  The virtual methods {\sf eval} and {\sf print} provide
for evaluation and printing of values.  The {\sf eval} method takes as
argument a target expression to which the evaluated expression will be
assigned, as well as two environments.  The first environment contains the
list of legal value-ops for the expression, while the second is the more
general environment in which the expression is to be evaluated.
The default method for {\sf eval} merely assigns the current expression to
the target.  This suffices for objects, such as integers, which yield
themselves no matter how many times they are evaluated.
The default method {\sf print}, on the other hand, prints an error message.
Thus this method should always be overridden in subclasses.

\begin{figure}
\begin{cprog}
class Expression {
private:
	friend class Expr;
	int referenceCount;
public:
	Expression();

	virtual void free();

	// basic object protocol
	virtual void eval(Expr &, Environment *, Environment *);
	virtual void print();

	// conversion tests
	virtual Expression * touch();
	virtual IntegerExpression * isInteger();
	virtual Symbol * isSymbol();
	virtual Function * isFunction();
	virtual ListNode * isList();
	virtual Environment * isEnvironment();
	virtual APLValue * isAPLValue();
	virtual Method * isMethod();
};
\end{cprog}
\caption{The Class {\sf Expression}}\label{expression}
\end{figure}

\subsubsection{Memory Management}

For long running programs it is imperative that memory associated with
unused expressions be recovered by the underlying operating system.
This is accomplished in these interpreters through the mechanism of
reference counts.  Every expression contains a reference count field,
which is initially set to zero by the constructor in class {\sf Expression}.
The integer value maintained in this field represents the number of
pointers that reference the object.  When this count becomes zero, no
pointers refer to the object and the memory associated with it can be
recovered.

The maintenance of reference counts if peformed by the class 
{\sf Expr} (Figure~\ref{expr}).
As with the class {\sf Env} we have already encountered, the class {\sf Expr}
is a holder class, which maintains an expression pointer.
A value can be inserted into an {\sf Expr} either through construction or
the assignment operator.  A value can be retrieved either though the
protected method {\sf val} or, as a notational convenience, through the 
parenthesis
operator.  The method {\sf evalAndPrint}, as have noted already, merely 
passes the {\sf eval} message on to the underlying expression and
prints the resulting value.

\begin{figure}
\begin{cprog}
class Expr {
private:
	Expression * value;

protected:
	Expression * val()
		{ return value; }

public:
	Expr(Expression * = 0);

	Expression * operator ()()
		{ return val(); }

	void operator = (Expression *);

	void evalAndPrint(Environment *, Environment *);
};
\end{cprog}
\caption{The class {\sf Expr}}\label{expr}
\end{figure}

Figure~\ref{assign} gives the implementation of the constructor and
assignment operator for class {\sf Expr}.  The constructor takes an
optional pointer to an expression, which may be a null expression (the
default).  If the expression is
non-null, the reference count for the expression is incremented.
Similarly, the assignment operator first increments the reference count of
the new expression.  Then it decrements the reference count of the existing
expression (if non-null), and if the reference count reaches zero,
the memory is released, using the system function {\sf delete}.
Immediately prior to destruction, the virtual method {\sf free} is invoked.
Classes can override this method to provide any necessary class-specific 
maintenance.  For example, the class {\sf Environment} (Figure~\ref{environment})
assigns null values to the structures {\sf theNames}, {\sf theValues} and
{\sf theLink}, thereby possibly triggering the release of their storage as
well.

\begin{figure}
\begin{cprog}
Expr::Expr(Expression * val)
{
	value = val;
	if (val) value->referenceCount++;
}

void Expr::operator = (Expression * newvalue)
{
	// increment right hand side of assignment
	if (newvalue) {
		newvalue->referenceCount++;
		}

	// decrement left hand side of assignment 
	if (value) {
		value->referenceCount--;
		if (value->referenceCount = 0) {
			value->free();
			delete value;
			}
		}

	// then do the assignment
	value = newvalue;
}
\end{cprog}
\label{Assignment and Initialization of Expressions}\label{assign}
\end{figure}

\subsubsection{Type Conversion}

A common difficulty in a statically typed language such as C++ is the {\em
container problem}.  Elements placed into a general purpose data structure, such
as a list, must have a known type.  Generally this is accomplished by
declaring such elements as a general type, such as {\sf Expression}.  But
in reality such elements are usually instances of a more specific subclass,
such as an integer or a symbol.  When we remove these values from the list,
we would like to be able to recover the original type.

There are actually two steps in the solution of this problem.
The first step is testing the type of an object, to see if it is of a
certain form.   The second step is to legally assign the object to 
a variable declared as the more specific class.  In these interpreters
the mechanism of virtual methods is used to combine these two functions.
In the abstract class {\sf Expression} a number of virtual functions are
defined, such as {\sf isInteger} and {\sf isEnvironment}.   These are
declared as returning a pointer type.   The default behavior, as provided
by class {\sf Expression}, is to return a null pointer.  In an appropriate
class, however, this method is overridden so as to return the current element.
That is, the class associated with integers overrides {\sf isInteger}, the
class associated with symbols overrides {\sf isSymbol}, and so on.
Figure~\ref{isEnvironment} shows the two definitions of {\sf
isEnvironment}, the first from class {\sf Expression} and the second from
class {\sf Environment}.  
By testing whether the result of this method is non-null or not, one can
not only test the type of an object but one can assign the value to
a specific class pointer without compromising type safety.
An example bit of code is provided in Figure~\ref{isEnvironment} that
illustrates the use of these functions.

\begin{figure}
\begin{cprog}
Environment * Expression::isEnvironment() 
{ 
	return 0; 
}

Environment * Environment::isEnvironment()
{	
	return this; 
}

Expression * a = new Symbol("test");
Expression * b = new Environment(emptyList, emptyList, 0);

Environment * c = a->isEnvironment();	// will yield null
Environment * d = b->isEnvironment();	// will yield the environment

if (c) 
	printf("c is an environment");	// won't happen
if (d)
	printf("d is an environment");	// will happen
\end{cprog}
\caption{Type safe object test and conversion}\label{isEnvironment}
\end{figure}

The method {\sf touch} presents a slightly different situation.
It is defined in the abstract class to merely return the object to which
the message is sent.  That is, it is a null-operation.  In
Chapter~\ref{sasl}, when we introduce delayed evaluation, we will define a
type of expression which is not evaluated until it is needed.   This
expression will override the touch method to force evaluation at that point.

\subsection{Integers}

Internally within the interpreters integers are represented by the class
{\sf IntegerExpression} (Figure~\ref{integerexpr}).  The actual integer
value is maintained as a private value set as part of the construction
process.  This value can be accessed via the method {\sf val}.
The only overridden methods are the {\sf print} method, which prints the
integer value, and the {\sf isInteger} method, which yields the current
object.

\begin{figure}
\begin{cprog}
class IntegerExpression : public Expression {
private:
	int value;
public:
	IntegerExpression(int v) 
		{ value = v; }

	virtual void print();
	virtual IntegerExpression * isInteger();

	int val()
		{ return value; }
};
\end{cprog}
\caption{The class {\sf IntegerExpression}}\label{integerexpr}
\end{figure}

\subsection{Symbols}

Symbols are used to represent uninterpreted character strings, for example
identifier names.  Instances of class {\sf Symbol} (Figure~\ref{symbol})
maintain the text of their value in a private instance variable.  This
character pointer can be recovered via the method {\sf chars}.
Storage for this text is allocated as part of the construction process,
and deleted by the virtual method {\sf free}.  The equality testing
operators return true if the current symbol matches the text of the
argument.

\begin{figure}
\begin{cprog}
class Symbol : public Expression {
private:
	char * text;

public:
	Symbol(char *);

	virtual void free();
	virtual void eval(Expr &, Environment *, Environment *);
	virtual void print();
	virtual Symbol * isSymbol();

	int operator == (Expression *);
	int operator == (char *);
	char * chars() { return text; }
};

void Symbol::eval(Expr & target, Environment * valueops, Environment * rho)
{
	Expression * result = rho->lookup(this);
	if (result)
		result = result->touch();
	else
		result = error("evaluation of unknown symbol: ", text);
	target = result;
}
\end{cprog}
\caption{The class {\sf Symbol}}\label{symbol}
\end{figure}

Figure~\ref{symbol} also shows the implementation of the method {\sf eval}
in the class {\sf Symbol}.  When a symbol is evaluated it is used as a key
to index the current environment.  If found the (possibly touched)
associated value is assigned to the target.  If it is not found an error 
message is generated.   The routine {\sf error} always yields a null 
expression.

\subsection{Lists}\label{listsec}

We have already encountered the behavior of the List data structure
(Figure~\ref{list}) in the discussion of environments.
As with expressions and environments, lists are represented by a pair of
classes.  The first, class {\sf ListNode}, maintains the actual list data.
The second, class {\sf List}, is merely a pointer to a list node, and
exists only to provide memory management operations.

Only one feature of the latter class deserves comment; rather than
overloading the parenthesis operator the class {\sf List} defines a
conversion operator which permits instances of class {\sf List} to be
converted without comment to {\sf ListNodes}.  Thus in most cases a {\sf
List} can be used where a {\sf ListNode} is expected, and the conversion
will be implicitly defined.  We have seen this already, without having
noted the fact, in
several places where the variable {\sf emptyList} (an instance of class
{\sf List}) was used in situations where an instance of class {\sf
ListNode} was required.

\begin{figure}
\begin{cprog}
class ListNode : public Expression {
protected:
	Expr h;		// the head field
	Expr t;		// the tail field

public:
	ListNode(Expression *, Expression *);

	// overridden methods
	virtual void free();
	virtual void eval(Expr &, Environment *, Environment *);
	virtual void print();
	virtual ListNode * isList();

	// list specific methods
	int isNil();
	int length();
	Expression * at(int);
	virtual Expression * head();
	void head(Expression * x);
	ListNode * tail();
};


class List : public Expr {
public:
	operator ListNode *();
	void operator = (ListNode * r);
};
\end{cprog}
\caption{The classes {\sf List} and {\sf ListNode}}\label{list}
\end{figure}

The actual list data is maintained in the instance variables {\sf h} and
{\sf t}, which we have already noted can be retrieved (and, in the case of
the h, set) by the methods {\sf head} and {\sf tail}.  The method {\sf
length} returns the length of a list, and the method {\sf at} permits a
list to be indexed as an array, starting with zero for the head position.

The majority of methods, such as {\sf length}, {\sf at}, {\sf print},
are simple recursive routines, and will not be discussed.  Only one method
is sufficiently complex to deserve comment, and this is the procedure used to
evaluate a list.  A list is interpreted as a function call, and thus 
the evaluation of a list involves finding the indicated function and
invoking it, passing as arguments the remainder of the list.  These actions
are performed by the method {\sf eval} shown in Figure~\ref{listeval}.
An empty list always evaluates to itself.  Otherwise the first argument to
the list is examined.  If it is a symbol, a test is performed to see if it
is one of the value-ops.  If it is not found on the value-op list the first
element is evaluated, whether or not it is a symbol.  Generally this will
yield a function value.  If so, the method {\sf apply}, which we will
discuss in the next section, is used to invoke the function.
If the first argument did not evaluate to a function an error is indicated.

\begin{figure}
\begin{cprog}
void ListNode::eval(Expr & target, Environment * valueops, Environment * rho)
{

	// an empty list evaluates to nil
	if (isNil()) {
		target = this;
		return;
		}

	// if first argument is a symbol, see if it is a valueop
	Expression * firstarg = head();
	Expression * fun = 0;
	Symbol * name = firstarg->isSymbol();
	if (name) 
		fun = valueops->lookup(name);

	// otherwise evaluate it in the given environment
	if (! fun)  {
		firstarg->eval(target, valueops, rho);
		fun = target();
		}

	// now see if it is a function
	Function * theFun = 0;
	if (fun) theFun = fun->isFunction();
	if (theFun) { 
		theFun->apply(target, tail(), rho);
		}
	else {
		target = error("evaluation of unknown function");
		return;
		}
}
\end{cprog}
\caption{The method {\sf eval} from class {\sf List}}\label{listeval}
\end{figure}

\section{Functions}

If expressions are the heart of the interpreter, then functions are the 
muscles that keep the heart working.  All behavior, statements, valueops,
as well as user-defined functions, are implemented as subclasses of 
class {\sf Function} (Figure~\ref{function}).
As we noted in the last section, when a function (written as a list
expression) is evaluated the method {\sf apply} in invoked.
This method takes as argument the target for the evaluation and a list of
unevaluated arguments.  The default behavior in class {\sf Function} is to
evaluate the arguments, using the simple recursive routine {\sf evalArgs},
then invoke the method {\sf applyWithArgs}.

\begin{figure}
\begin{cprog}
class Function : public Expression {
public:
	virtual Function * isFunction();

	virtual min.log
	
	void apply(Expr &, ListNode *, Environment *);
	virtual void applyWithArgs(Expr &, ListNode *, Environment *);
	virtual void print();

	// isClosure is recognized only by functions
	virtual int isClosure();
};

static ListNode * evalArgs(ListNode * args, Environment * rho)
{
	if (args->isNil())
		return args;
	Expr newhead;
	Expression * first = args->head();
	first->eval(newhead, valueOps, rho);
	return new ListNode(newhead(), evalArgs(args->tail(), rho));
	newhead = 0;
}

void Function::apply(Expr & target, ListNode * args, Environment * rho)
{
	List newargs = evalArgs(args, rho);
	applyWithArgs(target, newargs, rho);
	newargs = 0;
}
\end{cprog}
\caption{The class {\sf Function} and the method {\sf apply}}\label{function}
\end{figure}

Both the methods {\sf apply} and {\sf applyWithArgs} are declared as
virtual, and can thus be overridden in subclasses.  Those function that do
not evaluate their arguments, such as the functions implementing the
control structures of Chapter 1, override the {\sf apply} method.
Function that do evaluate their arguments, such as the majority of
value-Ops, override the {\sf applyWithArgs} method.

Two subclasses of {\sf Function} deserve mention.  The class 
{\sf UnaryFunction} overrides {\sf apply} to test that only one argument 
has been provided.  Similarly the class {\sf BinaryFunction} tests for
exactly two arguments.  
The remaining major subclass of {\sf Function} is the class {\sf
UserFunction}.  We will defer a discussion of this until we examine
the implementation of the {\sf define} statement.

\section{The Basic Evaluator}

We are now in a position to finally describe the characteristics that are
unique to the basic evaluator of chapter one.  This interpreter recognizes
one command (the {\sf define} statement), several built-in statements
({\sf if, while, set}, and {\sf begin}), and a number of value-ops.
All are implemented internally as functions.  What syntactic category a
symbol is associated with is
determined by what environment it is placed on, and not by the
structure of the function.

\subsection{The define statement}

The define statement is implemented as the single instance of the class
{\sf DefineStatement} (Figure~\ref{define}), entered with the key
``define'' in the {\sf commands} environment.  The class overrides the
virtual method {\sf apply}, since it must access its arguments before they
are evaluated.  It tests that the arguments are exactly three in number,
and that the first is a symbol and the second a list.
If no errors are detected, an instance of the class {\sf UserFunction} is
created and and set in the current (always global) environment.

\begin{figure}
\begin{cprog}
class DefineStatement : public Function {
public:
	virtual void apply(Expr &, ListNode *, Environment *);
};

void DefineStatement::apply(Expr & target, ListNode * args, Environment * rho)
{
	if (args->length() != 3) {
		target = error("define requires three arguments");
		return;
		}

	Symbol * name = args->at(0)->isSymbol();
	if (! name) {
		target = error("define missing name");
		return;
		}

	ListNode * argNames = args->at(1)->isList();
	if (! argNames) {
		target = error("define missing arg names list");
		return;
		}

	rho->set(name, new UserFunction(argNames, args->at(2), rho));

	// yield as value the name of the function
	target = name;
};
\end{cprog}
\caption{Implementation of the {\sf define} statement}\label{define}
\end{figure}

The class {\sf UserFunction} created by the define statement is similarly a
subclass of class {\sf Function} (Figure~\ref{userfunction}).
User functions maintain in instance variables the list of argument names, 
the body of the
function, and the lexical context in which they are to execute.
These values are set by the constructor when the function is defined,
and freed by the virtual method {\sf free} when no longer needed.

\begin{figure}
\begin{cprog}
class UserFunction : public Function {
protected:
	List argNames;
	Expr body;
	Env context;
public:
	UserFunction(ListNode *, Expression *, Environment *);
	virtual void free();
	virtual void applyWithArgs(Expr &, ListNode *, Environment *);
	virtual int isClosure();
};

void UserFunction::applyWithArgs(Expr& target, ListNode* args, Environment* rho)
{
	// number of args should match definition
	ListNode *an = argNames;
	if (an->length() != args->length()) {
		error("argument length mismatch");
		return;
		}

	// make new environment
	Env newrho; 
	newrho = new Environment(an, args, context);

	// evaluate body in new environment
	Expression * bod = body();
	if (bod)
		bod->eval(target, valueOps, newrho);
	else
		target = 0;

	newrho = 0;	// force memory recovery
}
\end{cprog}
\caption{The class {\sf UserFunction} and method of application}\label{userfunction}
\end{figure}

User functions always work with evaluated arguments, and thus they override
the method {\sf applyWithArgs}.   The implementation of this method is
also shown in Figure~\ref{userfunction}.  This method checks that the
number of arguments supplied matches the number in the function definition,
then creates a new environment to match the arguments and their values.
The expression which represents the body of the function is then evaluated.
By passing the new context as argument to the evaluation, symbolic
references to the arguments will be matched with the appropriate values.

\subsection{Built-In Statements}

The built-in statements {\sf if, while, set} and {\sf begin} are each
defined by functions entered in the {\sf valueOps} environment.
With the exception of {\sf begin}, these must capture their arguments
before they are evaluated and thus, like {\sf define}, they override the
method {\sf apply}.

\subsubsection{The If statement}

The If statement (Figure~\ref{ifstatement}) first insures it has three
arguments.  It then evaluates the first argument.  Using the auxiliary
function {\sf isTrue} (which will vary over the different interpreters as
our definition of ``true'' changes) the truth or falsity of the first
expression is determined.  Depending upon the outcome, either the second or
third argument is evaluated to determine the result.
In the Chapter 1 interpreter the value 0 is false, and all other values
(integer or not) are considered to be true.

\begin{figure}
\begin{cprog}
class IfStatement : public Function {
public:
	virtual void apply(Expr &, ListNode *, Environment *);
};

void IfStatement::apply(Expr & target, ListNode * args, Environment * rho)
{
	if (args->length() != 3) {
		target = error("if statement requires three arguments");
		return;
		}

	Expr cond;
	args->at(0)->eval(cond, valueOps, rho);
	if (isTrue(cond()))
		args->at(1)->eval(target, valueOps, rho);
	else
		args->at(2)->eval(target, valueOps, rho);
	cond = 0;
}

int isTrue(Expression * cond)
{
	IntegerExpression *ival = cond->isInteger();
	if (ival && ival->val() == 0)
		return 0;
	return 1;
}
\end{cprog}
\caption{The implementation of the If statement}\label{ifstatement}
\end{figure}

\subsubsection{The while statement}

The function that implements the while statement is shown in
Figure~\ref{whilestmt}.  Although the while statement requires two
arguments, it nevertheless cannot usefully be made a subclass of class {\sf
BinaryFunction}, since it must access its arguments before they are evaluated.
The implementation of the while statements loops until the first argument
evaluates to a true condition, using the same test for true method used by
the if statement.  The results returned by evaluating the body of the while
statement are ignored, as the body is executed just for side effects.

\begin{figure}
\begin{cprog}
void WhileStatement::apply(Expr & target, ListNode * args, Environment * rho)
{	Expr stmt;

	if (args->length() != 2) {
		target = error("while statement requires two arguments");
		return;
		}

	// grab the two pieces of the statement
	Expression * condexp = args->at(0);
	Expression * stexp = args->at(1);

	// then start the execution loop
	condexp->eval(target, valueOps, rho);
	while (isTrue(target())) {
		// evaluate body
		stexp->eval(stmt, valueOps, rho);
		// but ignore it (and force memory reclamation)
		stmt = 0;
		// then reevaluate condition
		condexp->eval(target, valueOps, rho);
		}
}
\end{cprog}
\caption{The implementation of the While statement}\label{whilestmt}
\end{figure}

\subsubsection{The set statement}

The implementation of the set statement is shown in
Figure~\ref{setstatement}.  The function insures the first argument is a
symbol, evaluates the second argument, then sets the binding of the symbol
to value in the current environment.

\begin{figure}
\begin{cprog}
void SetStatement::apply(Expr & target, ListNode * args, Environment * rho)
{
	if (args->length() != 2) {
		target = error("set statement requires two arguments");
		return;
		}

	// get the two parts
	Symbol * sym = args->at(0)->isSymbol();
	if (! sym) {
		target = error("set commands requires symbol for first arg");
		return;
		}

	// set target to value of second argument
	args->at(1)->eval(target, valueOps, rho);

	// set it in the environment
	rho->set(sym, target());
}
\end{cprog}
\caption{The implementation of the set statement}\label{setstatement}
\end{figure}

\subsubsection{The begin statement}

Unlike the other statements, the begin statement does what to evaluate each
of its arguments.  Thus it overrides the method {\sf applyWithArgs},
instead of the method {\sf apply}.  It merely assigns to the target
variable the value of the last expression (Figure~\ref{beginstatement}).

\begin{figure}
\begin{cprog}
void BeginStatement::applyWithArgs(Expr& target, ListNode* args, Environment* rho)
{
	int len = args->length();

	// yield as result the end of the list
	if (len < 1) 
		target = error("begin needs at least one statement");
	else
		target = args->at(len - 1);
}
\end{cprog}
\caption{The implementation of the begin statement}\label{beginstatement}
\end{figure}

\subsection{The value-Ops}

The Value-ops are functions placed in the {\sf valueop} global environment.
They can be divided into two categories; there are those that take two
integer arguments and produce an integer result ($+$, $-$, $*$, $/$, $=$,
$<$ and $>$) and those that take a single argument ({\sf print}).  

The implementation of the integer binary functions is simplified by the
introduction of an intermediate class {\sf IntegerBinaryFunction}, a
subclass of {\sf BinaryFunction} (Figure~\ref{plusfun}).  
The private state for each instance of this class holds a pointer to a
function that takes two integer values and generates an integer result.
The {\sf applyWithArgs} method in this class
decodes the two integer arguments, then invokes the stored function to
produce the new integer value.  To implement each of the seven binary
integer functions (the relational functions generate 0 and 1 values for
true and false, remember) it is only necessary define an appropriate
function and pass it as argument to the constructor during initialization of
the interpreter.  This can be seen in Figure~\ref{chap1init}.

The print function is implemented by a subclass of {\sf UnaryFunction}
that merely invokes the method {\sf print} on the argument.  All
expressions will respond to this method.

\begin{figure}
\begin{cprog}
class IntegerBinaryFunction : public BinaryFunction {
private:
	int (*fun)(int, int);
public:
	IntegerBinaryFunction(int (*afun)(int, int));
	virtual void applyWithArgs(Expr &, ListNode *, Environment *);
	virtual int value(int, int);
};

void IntegerBinaryFunction::applyWithArgs(Expr& target, ListNode  args, 
		Environment* rho)
{
	Expression * left = args->at(0);
	Expression * right = args->at(1);
	if ((! left->isInteger()) || (! right->isInteger())) {
		target = error("arithmetic function with nonint args");
		return;
		}
	
	target =  new IntegerExpression(
		fun(left->isInteger()->val(), right->isInteger()->val()));
}

int PlusFunction(int a, int b) { return a + b; }
int MinusFunction(int a, int b) { return a - b; }
int TimesFunction(int a, int b) { return a * b; }
int DivideFunction(int a, int b)
{
	if (b != 0)
		return a / b;
	error("division by zero");
	return 0;
}
int IntEqualFunction(int a, int b) { return a == b; }
int LessThanFunction(int a, int b) { return a < b; }
int GreaterThanFunction(int a, int b) { return a > b; }
\end{cprog}
\caption{Implementation of the Arithmetic Functions}\label{plusfun}
\end{figure}

\section{Initializing the Run-Time Environment}

Figure~\ref{chap1init} shows the initialization routine for the
interpreters of chapter one.  In chapter one there are no global variables
defined at the start of execution.  There is one command, the statement
{\sf define}, and a number of value-ops.

\begin{figure}
\begin{cprog}
initialize()
{
	// create the reader/parser
	reader = new Reader;

	// initialize the commands environment
	Environment * cmds = commands;
	cmds->add(new Symbol("define"), new DefineStatement);

	// initialize the value-ops environment
	Environment * vo = valueOps;
	vo->add(new Symbol("if"), new IfStatement);
	vo->add(new Symbol("while"), new WhileStatement);
	vo->add(new Symbol("set"), new SetStatement);
	vo->add(new Symbol("begin"), new BeginStatement);
	vo->add(new Symbol("+"), new IntegerBinaryFunction(PlusFunction));
	vo->add(new Symbol("-"), new IntegerBinaryFunction(MinusFunction));
	vo->add(new Symbol("*"), new IntegerBinaryFunction(TimesFunction));
	vo->add(new Symbol("/"), new IntegerBinaryFunction(DivideFunction));
	vo->add(new Symbol("="), new IntegerBinaryFunction(IntEqualFunction));
	vo->add(new Symbol("<"), new IntegerBinaryFunction(LessThanFunction));
	vo->add(new Symbol(">"), new IntegerBinaryFunction(GreaterThanFunction));
	vo->add(new Symbol("print"), new PrintFunction);
}
\end{cprog}
\caption{Initialization of the Basic Evaluator}\label{chap1init}
\end{figure}

\chapter{The Lisp Interpreter}

The interpreter for Lisp differs only slightly from that of Chapter one.
The reader/parser is modified so as to recognize quoted constants, 
two new global variables ({\sf T} and {\sf nil}) are added, and a
number of new value-ops are defined.  In all other respects it is the same.
Figure~\ref{chap2hier} shows the class hierarchy for the expression classes
added in chapter 2.

\setlength{\unitlength}{5mm}
\begin{figure}
\begin{picture}(16,5)(-4,-3)
\put(-3.5,0){\sf Expression}
\put(0,0.2){\line(1,0){1}}
\put(1,0){\sf Function}
\put(0,0.2){\line(1,-1){1}}
\put(1,-1){\sf QuotedConstant}
\put(4,0.2){\line(1,0){1}}
\put(5,0){\sf BinaryFunction}
\put(4,0.2){\line(1,3){1}}
\put(5,3){\sf UnaryFunction}
\put(9.5,3.2){\line(1,0){1}}
\put(10.5,3){\sf BooleanUnary}
\put(9.5,0.2){\line(1,1){1}}
\put(10.5,1){\sf IntegerBinaryFunction}
\put(9.5,0.2){\line(1,-1){1}}
\put(10.5,-1){\sf BooleanBinaryFunction}
\end{picture}
\caption{Classes added in Chapter Two}\label{chap2hier}
\end{figure}

\section{The Lisp reader}

The Lisp reader is created by subclassing from the base class {\sf
Reader} (Figure~\ref{lispreader}).  The only change is to modify the
method {\sf readExpression} to check for leading quote marks.  If no mark
is found, execution is as in the default case.  If a quote mark is found,
the character pointer is advanced and the following expression is turned
into a quoted constant.  Note that no checking is performed on this
expression.  This permits symbols, even separators, to be treated as data.
That is, '; is a quoted symbol, even though the semicolon itself is not a
legal symbol.

\begin{figure}
\begin{cprog}
class QuotedConst : public Expression {
private:
	Expr theValue;
public:
	QuotedConst(Expression * val)
		{ theValue = val; }

	virtual void free();
	virtual void eval(Expr &, Environment *, Environment *);
	virtual void print();
};

class LispReader : public Reader {
protected:
	virtual Expression * readExpression();
};

void QuotedConst::eval(Expr &target, Environment *, Environment *)
{
	target = theValue();
}

void QuotedConst::print()
{
	printf("'"); theValue()->print();
}

Expression * LispReader::readExpression()
{
	// if quoted constant, return it,
	if ((*p == '\'') || (*p == '`')) {
		p++;
		return new QuotedConst(readExpression());
		}
	// otherwise simply return what we had before
	return Reader::readExpression();
}
\end{cprog}
\caption{The Lisp reader/parser}\label{lispreader}
\end{figure}

To create quoted constants it is necessary to introduce a new type of
expression.  When an instance of class {\sf QuotedConst} is evaluated,
it simply returns its (unevaluated) data value.

\section{Value-ops}

In addition to adding a number of new value-ops, the Lisp interpreter
modifies the meaning of a few of the Chapter 1 functions.  For example
the relational operators must now return the values {\sf T} or {\sf nil},
and not 1 and 0 values.  Similarly the meaning of {\em true} and {\em false} 
used by the {\sf if} and {\sf while} statements is changed.  Finally the 
equality testing function ($=$) must now recognize both symbols and integers.

\subsection{Relationals}

Figure~\ref{equals} shows the revised definition of the equality testing
function, which now must be prepared to handle symbols and well as
integers.

\begin{figure}
\begin{cprog}
class EqualFunction : public BinaryFunction {
public:
	virtual void applyWithArgs(Expr&, ListNode *, Environment *);
};

void EqualFunction::applyWithArgs(Expr& target, ListNode * args, 
		Environment *rho)
{
	Expression * one = args->at(0);
	Expression * two = args->at(1);

	// true if both numbers and same number
	IntegerExpression * ione = one->isInteger();
	IntegerExpression * itwo = two->isInteger();
	if (ione && itwo && (ione->val() == itwo->val())) {
		target = true();
		return;
		}

	// or both symbols and same symbol
	Symbol * sone = one->isSymbol();
	Symbol * stwo = two->isSymbol();
	if (sone && stwo && (*sone == stwo)) {
		target = true();
		return;
		}

	// or both lists and both nil
	ListNode * lone = one->isList();
	ListNode * ltwo = two->isList();
	if (lone && ltwo && lone->isNil() && ltwo->isNil()) {
		target = true();
		return;
		}

	// false otherwise
	target = false();
}
\end{cprog}
\caption{The revised Definition of the equality function}\label{equals}
\end{figure}

Implementation of the boolean binary functions is simplified by the
introduction of a class {\sf BooleanBinaryFunction} (Figure~\ref{boolbin}).  
This class decodes
the two integer arguments and invokes a further method to determine the
boolean result.  Based on this result either the value of the global symbol
representing true or the symbol representing false is returned.

\begin{figure}
\begin{cprog}
class BooleanBinaryFunction : public BinaryFunction {
private:
	int (*fun)(int, int);
public:
	BooleanBinaryFunction(int (*thefun)(int, int)) { fun = thefun; }
	virtual void applyWithArgs(Expr&, ListNode*, Environment*);
	virtual int value(int, int);
};

void BooleanBinaryFunction::applyWithArgs(Expr& target, ListNode* args, 
	Environment* rho)
{
	Expression * left = args->at(0);
	Expression * right = args->at(1);
	if ((! left->isInteger()) || (! right->isInteger())) {
		target = error("arithmetic function with nonint args");
		return;
		}
	
	if (value(left->isInteger()->val(), right->isInteger()->val())) 
		target = true();
	else 
		target = false();
}

int LessThanFunction::value(int a, int b) { return a < b; }
int GreaterThanFunction::value(int a, int b) { return a > b; }

int isTrue(Expression * cond)
{
	// the only thing false is nil
	ListNode *nval = cond->isList();
	if (nval && nval->isNil())
		return 0;
	return 1;
}
\end{cprog}
\caption{Returning boolean results from relationals}\label{boolbin}
\end{figure}

Finally Figure~\ref{boolbin} shows the revised function used by if and
while statements to determine the truth or falsity of their condition.
Unlike in Chapter 1, where 0 and 1 were used to represent true and false,
here {\sf nil} is used as the only false value.

\subsection{Car, Cdr and Cons}

\begin{figure}
\begin{cprog}
void CarFunction(Expr & target, Expression * arg)
{
	ListNode * thelist = arg->isList();
	if (! thelist) {
		target = error("car applied to non list");
		return;
		}
	target = thelist->head()->touch();
}

void CdrFunction(Expr & target, Expression * arg)
{
	ListNode * thelist = arg->isList();
	if (! thelist) {
		target = error("car applied to non list");
		return;
		}
	target = thelist->tail()->touch();
}

void ConsFunction(Expr & target, Expression * left, Expression * right)
{
	target = new ListNode(left, right);
}
\end{cprog}
\caption{Implementation of Car, Cdr and Cons}\label{car}
\end{figure}

Car and cdr are implemented as simple unary functions (Figure~\ref{car}),
and cons is a simple binary function that creates a new {\sf ListNode} out
of its two arguments.\footnote{A matter for debate is whether Cons should
give an error if the second argument is not a list.  Real Lisp doesn't
care; but also uses a different format for printing such lists.  Our
interpreter prints such as lists exactly as if the second argument had been
a list containing the element.}

\subsection{Predicates}

The implementation of the predicates {\sf number?}, {\sf symbol?}, {\sf
list?} and {\sf null?} is simplified by the creation of a class {\sf
BooleanUnary} (Figure~\ref{boolunary}), subclassing {\sf UnaryFunction}.  
As with the integer functions implemented in chapter 1, instances of 
{\sf BooleanUnary} maintain as part of their state a function that takes an
expression and returns an integer (that is, boolean) value.
Thus for each predicate it is only necessary to write a function which takes 
the single argument and returns a true/false indication.

\begin{figure}
\begin{cprog}
class BooleanUnary : public UnaryFunction {
private:
	int (*fun)(Expression *);
public:
	BooleanUnary(int (*thefun)(Expression *);
	virtual void applyWithArgs(Expr& target, ListNode* args, Environment*);
};

void BooleanUnary::applyWithArgs(Expr & target, ListNode * args, Environment *)
{
	if (fun(args->head()))
		target = true();
	else
		target = false();
}

int NumberpFunction(Expression * arg)
{
	return 0 != arg->isInteger();
}

int SymbolpFunction(Expression * arg)
{
	return 0 != arg->isSymbol();
}

int ListpFunction(Expression * arg)
{
	ListNode * x = arg->isList();
	// list? doesn't return true on nil
	if (x && x->isNil()) return 0;
	if (x) return 1;
	return 0;
}

int NullpFunction(Expression * arg)
{
	ListNode * x = arg->isList();
	return x && x->isNil();
}
\end{cprog}
\caption{The class Boolean Unary}\label{boolunary}
\end{figure}

\section{Initialization of the Lisp Interpreter}

Figure~\ref{chap2init} shows the initialization method for the Lisp
interpreter.

\begin{figure}
\begin{cprog}
initialize()
{

	// create the reader/parser 
	reader = new LispReader;

	// initialize the global environment
	Symbol * truesym = new Symbol("T");
	true = truesym;
	false = emptyList();
	Environment * genv = globalEnvironment;
	// make T evaluate to T always
	genv->add(truesym, truesym);
	genv->add(new Symbol("nil"), emptyList());

	// initialize the commands environment
	Environment * cmds = commands;
	cmds->add(new Symbol("define"), new DefineStatement);

	// initialize the value-ops environment
	Environment * vo = valueOps;
	vo->add(new Symbol("if"), new IfStatement);
	vo->add(new Symbol("while"), new WhileStatement);
	vo->add(new Symbol("set"), new SetStatement);
	vo->add(new Symbol("begin"), new BeginStatement);
	vo->add(new Symbol("+"), new IntegerBinaryFunction(PlusFunction));
	vo->add(new Symbol("-"), new IntegerBinaryFunction(MinusFunction));
	vo->add(new Symbol("*"), new IntegerBinaryFunction(TimesFunction));
	vo->add(new Symbol("/"), new IntegerBinaryFunction(DivideFunction));
	vo->add(new Symbol("="), new BinaryFunction(EqualFunction));
	vo->add(new Symbol("<"), new BooleanBinaryFunction(LessThanFunction));
	vo->add(new Symbol(">"), new BooleanBinaryFunction(GreaterThanFunction));
	vo->add(new Symbol("cons"), new BinaryFunction(ConsFunction));
	vo->add(new Symbol("car"), new UnaryFunction(CarFunction));
	vo->add(new Symbol("cdr"), new UnaryFunction(CdrFunction));
	vo->add(new Symbol("number?"), new BooleanUnary(NumberpFunction));
	vo->add(new Symbol("symbol?"), new BooleanUnary(SymbolpFunction));
	vo->add(new Symbol("list?"), new BooleanUnary(ListpFunction));
	vo->add(new Symbol("null?"), new BooleanUnary(NullpFunction));
	vo->add(new Symbol("print"), new UnaryFunction(PrintFunction));
}
\end{cprog}
\caption{Initialization of the Lisp interpreter}\label{chap2init}
\end{figure}

\chapter{The APL Interpreter}

My version of the APL interpreter differs somewhat from that provided by
Kamin:
\begin{itemize}
\item
My version will recognize arbitrary rank (dimension) arrays, not simply
scalar, vector and two dimensional arrays.  (Although currently it is only
able to print those three types).
\item
The C++ version of the interpreter recognizes vector constants without 
the necessity for quoting them, as in (resize (3 4) (indx 12)).
\item
I have eliminated the if and while statements, thus forcing programmers
into a more ``APL'' style of thought.
\item
My version of catenation works now for values of arbitrary dimensionality.
(Transpose and print are the only two functions that limit the
dimensionality of their arguments).
\end{itemize}

Despite the APL interpreter being larger than any other interpreter, I
think that the addition of a few more functions could give the student
an even better feel for the language, as well as providing a smooth
transition to functional programming.  Specifically, I think reduction
should be defined as a functional, and inner and outer products added as
operations.  I have not done this as yet, however.

Figure~\ref{aplhier} shows the class hierarchy for the classes introducted
in this chapter.

\setlength{\unitlength}{5mm}
\begin{figure}
\begin{picture}(25,10)(-4,-5)
\put(-3.5,0){\sf Expression}
\put(0,0.2){\line(1,0){1}}
\put(1,0){\sf Function}
\put(0,0.2){\line(1,-2){1}}
\put(1,-2){\sf APLValue}
\put(4,0.2){\line(1,-2){1}}
\put(5,-2){\sf BinaryFunction}
\put(9.5,-1.8){\line(1,0){1}}
\put(10.5,-2){\sf APLBinaryFunction}
\put(4,0.2){\line(1,2){1}}
\put(5,2){\sf UnaryFunction}
\put(9.5,2.2){\line(1,0){1}}
\put(10.5,2){\sf APLUnaryFunction}
\put(16,2.2){\line(1,0){1}}
\put(17,2){\sf RavelFunction}
\put(16,2.2){\line(1,1){1}}
\put(17,3){\sf ShapeFunction}
\put(16,2.2){\line(1,2){1}}
\put(17,4){\sf APLReduction}
\put(16,2.2){\line(1,-1){1}}
\put(17,1){\sf IndexFunction}
\put(16,2.2){\line(1,3){1}}
\put(17,5){\sf TransposeFunction}
\put(16,-1.8){\line(1,0){1}}
\put(17,-2){\sf RestructFunction}
\put(16,-1.8){\line(1,1){1}}
\put(17,-1){\sf CompressFunction}
\put(16,-1.8){\line(1,-1){1}}
\put(17,-3){\sf APLScalarFunction}
\put(16,-1.8){\line(1,-2){1}}
\put(17,-4){\sf CatenationFunction}
\put(16,-1.8){\line(1,-3){1}}
\put(17,-5){\sf SubscriptionFunction}
\end{picture}
\caption{The APL interpreter class hierarchy}\label{aplhier}
\end{figure}

\section{APL Values}

The APL interpreter manipulates APL values, which are defined by the data
type {\sf APLValue} (Figure~\ref{aplvalue}).  An APL value represents a
integer rectilinear array.  Internally, such a value is represented by a
list that maintains the shape (extent along each dimension) and a vector of
integer values.  The length of the shape list provides the rank
(dimensionality) of the data value.  The product of the values in the shape
indicates the number of elements in the array, except in the case of scalar
values, which have an empty shape array.

\begin{figure}
\begin{cprog}
class APLValue : public Expression {
private:
	List shapedata;
	int * data;
public:
	APLValue(ListNode *, int);

	// the overridden methods
	virtual APLValue * isAPLValue();
	virtual void free();
	virtual void print();

	// methods unique to apl values
	int size();
	ListNode * shape();
	int shapeAt(int);
	int at(int pos);
	void atPut(int pos, int val);
};
\end{cprog}
\caption{The Representation for APL Values}\label{aplvalue}
\end{figure}

APL values are stored in what is called {\em ravel-order}.  This is what in
some other languages is called row-major order.

The methods defined for APL values can be used to determine the number of
elements contained in the structure ({\sf size}), obtain the shape of the
value ({\sf shape}), obtain the shape at any given dimension 
({\sf shapeAt}), obtain the value at any given ravel-order position 
({\sf at}), and finally change the value at any position ({\sf atPut}).

\section{The APL Reader}

The APL reader is modified so that individual scalar values and vectors of
integers are recognized as APL values.
The class definition for {\sf APLreader} is shown in
Figure~\ref{aplreader}, and the code for the two auxiliary functions in the
next figure.

\begin{figure}
\begin{cprog}
class APLreader : public LispReader {
protected:
	virtual Expression * readExpression();
private:
	APLValue * readAPLscalar(int);
	APLValue * readAPLvector(int);
};

Expression * APLreader::readExpression()
{
	// see if it is a scalar value
	if ((*p == '-') && isdigit(*(p+1))) {
		p++;
		return readAPLscalar( - readInteger());
		}

	if (isdigit(*p))
		return readAPLscalar(readInteger());

	// see if it is a vector constant
	if (*p == '(') {
		p++; 
		skipNewlines();
		if (isdigit(*p))
			return readAPLvector(0);
		return readList();
		}

	// else default
	return LispReader::readExpression();
}
\end{cprog}
\caption{The APL reader}\label{aplreader}
\end{figure}

\begin{figure}
\begin{cprog}
APLValue * APLreader::readAPLscalar(int d)
{
	// read a scalar value, but make it an APL value
	APLValue * newval = new APLValue(emptyList, 1);
	newval->atPut(0, d);
	return newval;
}

APLValue * APLreader::readAPLvector(int size)
{
	skipNewlines();

	// if at end of list, make new vector
	if (*p == ')') {
		p++;
		return new APLValue(
			new ListNode(new IntegerExpression(size), emptyList()),
			size);
		}

	// else we better have a digit, save it and get the rest
	int sign = 1;
	if (*p == '-') { sign = -1; p++; }
	if (! isdigit(*p))
		error("ill formed apl vector constant");
	int val = sign * readInteger();
	APLValue * newval = readAPLvector(size + 1);
	newval->atPut(size, val);
	return newval;
}
\end{cprog}
\caption{The APL reader functions}\label{readvector}
\end{figure}

\section{APL Functions}

The implementation of the APL functions is simplified by the addition of
two auxiliary classes, {\sf APLUnary} and {\sf APLBinary}.   In addition
to checking that the right number of arguments are provided to a function
application, these check to insure that the arguments are APL
values\footnote{A largely gratuitous move, since the user has no way of
creating anything other than an APL value.  Still, it doesn't do any harm
to be careful.} and invoke yet another virtual function {\sf applyOp}, to
perform the actual calculation.

\subsection{Scalar Functions}

By far the largest class of APL functions are the so-called {\em scalar
functions}.  These are the conventional arithmetic and logical functions,
such as addition and multiplication, extended in the natural way to arrays.
The only complication in the implementation of these values concerns what
is called {\em scalar extension}.  That is, a scalar value can be used as
either the left or right argument to a scalar function, and it is treated
as if it were an entire array of the correct dimensionality to match the
other argument.  Since scalar extension can occur with either the left or
right argument, the code for scalar functions divides naturally into three
cases.

Scalar functions are implemented using a single class by making use, as we
have done before, of an instance variable that contains a pointer to 
a integer function that generates an integer result.  The class {\sf
APLscalarFunction} and the method {\sf applyOp} are shown in
Figure~\ref{aplscalar}.  Note that the same functions used in the previous
interpreters can be used in the construction of the APL scalar functions.

\begin{figure}
\begin{cprog}
void APLScalarFunction::applyOp(Expr& target, APLValue* left, APLValue* right)
{
	if (left->size() == 1) {	// scalar extension of left
		int extent = right->size();
		APLValue * newval = new APLValue(right->shape(), extent);
		int lvalue = left->at(0);
		while (--extent >= 0)
			newval->atPut(extent, fun(lvalue, right->at(extent)));
		target = newval;
		}
	else if (right->size() == 1) {	// scalar extension of right
		int extent = left->size();
		APLValue * newval = new APLValue(left->shape(), extent);
		int rvalue = right->at(0);
		while (--extent >= 0)
			newval->atPut(extent, fun(left->at(extent), rvalue));
		target = newval;
		}
	else {				// conforming arrays
		int extent = left->size();
		if (extent != right->size()) {
			target = error("conformance error on scalar function");
			return;
			}
		for (int i = left->shape()->length(); --i >= 0; ) 
			if (left->shapeAt(i) != right->shapeAt(i)) {
				target = 
				error("conformance error on scalar function");
				return;
				}

		APLValue * newval = new APLValue(left->shape(), extent);
		while (--extent >= 0)
			newval->atPut(extent, 
				fun(left->at(extent), right->at(extent)));
		target = newval;
		}
}
\end{cprog}
\caption{APL Scalar Functions}\label{aplscalar}
\end{figure}

\subsection{Reduction}

For each scalar function there is an associated reduction
function.\footnote{The statement is true of real APL.  The Kamin
interpreters do not implement reductions with relational operators, which
are, however, not particularly useful.}  Reduction in these interpreters always
occurs along the last dimension.  Thus to compute the size of a new value
is suffices to remove the last dimension value.  This also simplifies the
generation of the new values, since the argument array can be processed in
units as long as the final dimension.  As with the scalar functions,
there is one class defined for all the reductions, with each instance of
this class maintaining the particular scalar function being used for the
reduction operations.  Figure~\ref{reduction} shows the code used in
computing the APL reduction function.

\begin{figure}
\begin{cprog}
static int lastSize(ListNode * sz)
{
	int i = sz->length();
	if (i > 0) {
		IntegerExpression * ie = sz->at(i-1)->isInteger();
		if (ie)
			return ie->val();
		}
	return 1;
}

static ListNode * removeLast(ListNode * sz)
{
	ListNode * newsz = emptyList;
	int i = sz->length()-1;
	while (--i >= 0)
		newsz = new ListNode(sz->at(i), newsz);
	return newsz;
}

void APLReduction::applyOp(Expr & target, APLValue * arg)
{
	// compute the size of the new expression
	int rowextent = lastSize(arg->shape());
	int extent = arg->size() / rowextent;
	APLValue * newval = new APLValue(removeLast(arg->shape()), extent);

	while (--extent >= 0) {
		int start = (extent + 1) * rowextent - 1;
		int newint = arg->at(start);
		for (int i = rowextent - 2; i >= 0; i--)
			newint = fun(arg->at(--start), newint);
		newval->atPut(extent, newint);
		}

	target = newval;
}
\end{cprog}
\caption{Implementation of the APL reduction function}\label{reduction}
\end{figure}

\subsection{Compression}

Compression, like reduction, operates on the last dimension of a higher
order array, changing its extent to that of the number of one elements in
the left-argument vector.  The length of the left argument vector must
match the extent of the last dimension of the right argument.
The compression function (Figure~\ref{compress}) first computes the number
of one elements in the left argument, then iterates over the right argument
generating the new values.

\begin{figure}
\begin{cprog}
static ListNode * replaceLast(ListNode * sz, int i)
{
	ListNode *nz = new ListNode(new IntegerExpression(i), emptyList());
	for (i = sz->length() - 1; --i >= 0; )
		nz = new ListNode(sz->at(i), nz);
	return nz;
}

void CompressionFunction::applyOp(Expr& target, APLValue* left, APLValue* right)
{
	if (left->shape()->length() >= 2) {
		target = error("compression requires vector left arg");
		return;
		}
	int lsize = left->size();	// works for both scalar and vec
	int rsize = lastSize(right->shape());
	if (lsize != rsize) {
		target = error("compression conformability error");
		return;
		}
	// compute the number of non-zero values
	int i, nsize;
	nsize = 0;
	for (i = 0; i < lsize; i++)
		if (left->at(i)) nsize++;
	
	// now compute the new size
	int rextent = right->size();
	int extent = (rextent / lsize) * nsize;

	APLValue * newval = new APLValue(replaceLast(right->shape(), nsize),
				extent);

	// now fill in the values
	int index = 0;
	for (i = 0; i <= rextent; i++)
		if (left->at(i % lsize))
			newval->atPut(index++, right->at(i));
	target = newval;
}
\end{cprog}
\caption{The Compression function}\label{compress}
\end{figure}

\subsection{Shape and Reshape}

The {\sf shape} function merely copies the size on its argument into a new
APL value.  The reshape function ({\sf restruct}) generates a new value
with a size given by the left argument, which must be a vector, using
elements from the right argument, recycling over the ravel ordering of the
right argument multiple times if necessary.  The implementation of these
functions is shown in Figure~\ref{shape}.

\begin{figure}
\begin{cprog}
void ShapeFunction::applyOp(Expr & target, APLValue * arg)
{
	int extent = arg->shape()->length();
	ListNode * newshape = new ListNode(new IntegerExpression(extent),
			emptyList());
	APLValue * newval = new APLValue(newshape, extent);
	while (--extent >= 0) {
		IntegerExpression * ie = arg->shape()->at(extent)->isInteger();
		if (ie)
			newval->atPut(extent, ie->val());
		else
			target = error("impossible case in Shapefunction");
		}
	target = newval;
};

void RestructFunction::applyOp(Expr & target, APLValue * left, APLValue * right)
{
	int llen = left->shape()->length();
	if (llen >= 2) {
		target = error("restruct requires vector left arg");
		return;
		}
	llen = left->size();	// works for either scalar or vector
	int extent = 1;
	ListNode * newShape = emptyList;
	while (--llen >= 0) {
		newShape = new ListNode(new IntegerExpression(left->at(llen)),
			newShape);
		extent *= left->at(llen);
		}
	APLValue * newval = new APLValue(newShape, extent);
	int rsize = right->size();
	while (--extent >= 0)
		newval->atPut(extent, right->at(extent % rsize));
	target = newval;
}
\end{cprog}
\caption{The shape and reshape functions}\label{shape}
\end{figure}

\subsection{Ravel and Index}

The ravel function (Figure~\ref{ravel}) merely takes an argument of
arbitrary dimensionality and returns the values as a vector.  The index
function (called iota in real APL) takes a scalar value and returns a
vector of numbers from 1 to the argument value.

\begin{figure}
\begin{cprog}
void RavelFunction::applyOp(Expr & target, APLValue * arg)
{
	int extent = arg->size();
	APLValue * newval = new APLValue(extent);
	while (--extent >= 0) 
		newval->atPut(extent, arg->at(extent));
	target = newval;
}

void IndexFunction::applyOp(Expr & target, APLValue * arg)
{
	if (arg->size() != 1) {
		target = error("index function requires scalar argument");
		return;
		}
	int extent = arg->at(0);
	APLValue * newval = new APLValue(extent);
	while (--extent >= 0)
		newval->atPut(extent, extent + 1);
	target = newval;
}
\end{cprog}
\caption{Ravel and Index}\label{ravel}
\end{figure}

\subsection{Catenation}

The catenation function joins two arrays along their last dimension.
They must match in all other dimensions.  To build the new result first a
row from the first array is copies into the final array, then a row from
the second array, then another row from the first, followed by another row
from the second, and so on until all rows from each argument have been
used.

\begin{figure}
\begin{cprog}
void CatenationFunction::applyOp(Expr& target, APLValue* left, APLValue* right)
{
	ListNode * lshape = left->shape();
	ListNode * rshape = right->shape();
	int llen = lshape->length();
	int rlen = rshape->length();
	if (llen <= 0 || (llen != rlen)) {
		target = error("catenation conformability error");
		return;
		}

	// get the size of the last row in each structure
	int lrow, rrow;
	IntegerExpression * ie = lshape->at(llen-1)->isInteger();
	if (ie)
		lrow = ie->val();
	else
		lrow = 1;
	ie = rshape->at(rlen-1)->isInteger();
	if (ie)
		rrow = ie->val();
	else
		rrow = 1;

	// build up the new size
	int extent = lrow + rrow;
	ListNode * newShape = new ListNode(
		new IntegerExpression(extent), emptyList());
	llen = llen - 1;
	while (--llen >= 0) {
		newShape = new ListNode(lshape->at(llen), newShape);
		ie = lshape->at(llen)->isInteger();
		if (ie)
			extent *= ie->val();
		}

	APLValue * newval = new APLValue(newShape, extent);

	// now build the new values
	int i, index, lindex, rindex;
	index = lindex = rindex = 0;
	while (index < extent) {
		for (i = 0; i < lrow; i++)
			newval->atPut(index++, left->at(lindex++));
		for (i = 0; i < rrow; i++)
			newval->atPut(index++, right->at(rindex++));
		}

	target = newval;
}
\end{cprog}
\caption{Implementation of the Catenation function}\label{catenation}
\end{figure}

\subsection{Transpose}

While real APL defines transpose for arbitrary dimension arrays, the
transpose presented here works only for arrays of dimension two or less.
For vector and scalars the transpose does nothing.  Thus the only code
required (Figure~\ref{transpose}) is to take the transpose of a two
dimensional array.

\begin{figure}
\begin{cprog}
void TransposeFunction::applyOp(Expr& target, APLValue * arg)
{
	// transpose of vectors or scalars does nothings
	if (arg->shape()->length() != 2) {
		target = arg;
		return;
		}

	// get the two extents
	int lim1 = arg->shapeAt(0);
	int lim2 = arg->shapeAt(1);

	// build new shapes
	ListNode * newShape =
		new ListNode(arg->shape()->at(1),
		new ListNode(arg->shape()->at(0), emptyList()));
	APLValue * newval = new APLValue(newShape, lim1 * lim2);

	// now compute the values
	for (int i = 0; i < lim2; i++)
		for (int j = 0; j < lim2; j++)
			newval->atPut(i * lim1 + j,
				arg->at(j * lim2 + i));

	target = newval;
}
\end{cprog}
\caption{The Transpose Function}\label{transpose}
\end{figure}

\subsection{Subscription}

The Pascal interpreter provided by Kamin applies subscription to the first
dimension of a multidimension value.  In order to be consistent with the
other functions, my version does subscription along the last dimension.
Neither is exactly the same as the real APL version.  The subscription code
is shown in Figure~\ref{subscript}.

\begin{figure}
\begin{cprog}
void SubscriptFunction::applyOp(Expr& target, APLValue *left, APLValue *right)
{
	if (right->shape()->length() >= 2) {
		target = error("subscript requires vector second arg");
		return;
		}
	int rsize = right->size();
	int lsize = lastSize(left->shape());
	int extent = (left->size() / lsize) * rsize;

	APLValue * newval = new APLValue(replaceLast(left->shape(), rsize),
		extent);

	for (int i = 0; i < extent; i++)
		newval->atPut(i, left->at(
			(i / rsize) * lsize + (right->at(i % rsize)-1)));
	target = newval;
}
\end{cprog}
\caption{The Subscription function}\label{subscript}
\end{figure}

\section{Initialization of the APL interpreter}

The initialization code for the APL interpreter is shown in
Figure~\ref{chap3init}.

\begin{figure}
\begin{cprog}
initialize()
{

	// initialize global variables
	reader = new APLreader;

	// initialize the statement environment
	Environment * cmds = commands;
	cmds->add(new Symbol("define"), new DefineStatement);

	// initialize the value ops environment
	Environment * vo = valueOps;
	vo->add(new Symbol("set"), new SetStatement);
	vo->add(new Symbol("+"), new APLScalarFunction(PlusFunction));
	vo->add(new Symbol("-"), new APLScalarFunction(MinusFunction));
	vo->add(new Symbol("*"), new APLScalarFunction(TimesFunction));
	vo->add(new Symbol("/"), new APLScalarFunction(DivideFunction));
	vo->add(new Symbol("max"), new APLScalarFunction(scalarMax));
	vo->add(new Symbol("or"), new APLScalarFunction(scalarOr));
	vo->add(new Symbol("and"), new APLScalarFunction(scalarAnd));
	vo->add(new Symbol("="), new APLScalarFunction(scalarEq));
	vo->add(new Symbol("<"), new APLScalarFunction(LessThanFunction));
	vo->add(new Symbol(">"), new APLScalarFunction(GreaterThanFunction));
	vo->add(new Symbol("+/"), new APLReduction(PlusFunction));
	vo->add(new Symbol("-/"), new APLReduction(MinusFunction));
	vo->add(new Symbol("*/"), new APLReduction(TimesFunction));
	vo->add(new Symbol("//"), new APLReduction(DivideFunction));
	vo->add(new Symbol("max/"), new APLReduction(scalarMax));
	vo->add(new Symbol("or/"), new APLReduction(scalarOr));
	vo->add(new Symbol("and/"), new APLReduction(scalarAnd));
	vo->add(new Symbol("compress"), new CompressionFunction);
	vo->add(new Symbol("shape"), new ShapeFunction);
	vo->add(new Symbol("ravel"), new RavelFunction);
	vo->add(new Symbol("restruct"), new RestructFunction);
	vo->add(new Symbol("cat"), new CatenationFunction);
	vo->add(new Symbol("indx"), new IndexFunction);
	vo->add(new Symbol("trans"), new TransposeFunction);
	vo->add(new Symbol("[]"), new SubscriptFunction);
	vo->add(new Symbol("print"), new UnaryFunction(PrintFunction));
}
\end{cprog}
\caption{APL interpreter initialization}\label{chap3init}
\end{figure}

\chapter{The Scheme Interpreter}

After all the code required to generate the APL interpreter of Chapter 3,
the Scheme interpreter is simplicity in itself.  Of course, this has more
to do with the similarity of Scheme to the basic Lisp interpreter of
Chapter 2 than with any differences between APL and Scheme.

To implement Scheme it is only necessary to provide an implementation of
the lambda function.  This is accomplished by the class {\sf Lambda}, shown
in Figure~\ref{lambda}.  The actual implementation of lambda uses the same
class UserFunction we have seen in previous chapters.

\begin{figure}
\begin{cprog}
class LambdaFunction : public Function {
public:
	virtual void apply(Expr &, ListNode *, Environment *);
};

void LambdaFunction::apply(Expr & target, ListNode * args, Environment * rho)
{
	if (args->length() != 2) {
		target = error("lambda requires two arguments");
		return;
		}

	ListNode * argNames = args->head()->isList();
	if (! argNames) {
		target = error("lambda requires list of argument names");
		return;
		}

	target = new UserFunction(argNames, args->at(1), rho);
}
\end{cprog}
\caption{The class Lambda}\label{lambda}
\end{figure}

Initialization of the Scheme interpreter differs slightly from the code
used to initialize the Lisp interpreter (Figure~\ref{schemeinit}).  The 
{\sf define} command is no longer recognized, having been replaced by the
{\sf set}/{\sf lambda} pair.  The built-in arithmetic functions are now
considred to be global symbols, and not value-ops.  Indeed, there are no
comands or value-ops in this language.

\begin{figure}
\begin{cprog}
initialize()
{

	// initialize global variables
	reader = new LispReader;

	// initialize the value of true
	Symbol * truesym = new Symbol("T");
	true = truesym;
	false = emptyList();

	// initialize the command environment
	// there are no command or value-ops as such in scheme

	// initialize the global environment
	Environment * ge = globalEnvironment;
	ge->add(new Symbol("if"), new IfStatement);
	ge->add(new Symbol("while"), new WhileStatement);
	ge->add(new Symbol("set"), new SetStatement);
	ge->add(new Symbol("begin"), new BeginStatement);
	ge->add(new Symbol("+"), new IntegerBinaryFunction(PlusFunction));
	ge->add(new Symbol("-"), new IntegerBinaryFunction(MinusFunction));
	ge->add(new Symbol("*"), new IntegerBinaryFunction(TimesFunction));
	ge->add(new Symbol("/"), new IntegerBinaryFunction(DivideFunction));
	ge->add(new Symbol("="), new BinaryFunction(EqualFunction));
	ge->add(new Symbol("<"), new BooleanBinaryFunction(LessThanFunction));
	ge->add(new Symbol(">"), new BooleanBinaryFunction(GreaterThanFunction));
	ge->add(new Symbol("cons"), new BinaryFunction(ConsFunction));
	ge->add(new Symbol("car"), new UnaryFunction(CarFunction));
	ge->add(new Symbol("cdr"), new UnaryFunction(CdrFunction));
	ge->add(new Symbol("number?"), new BooleanUnary(NumberpFunction));
	ge->add(new Symbol("symbol?"), new BooleanUnary(SymbolpFunction));
	ge->add(new Symbol("list?"), new BooleanUnary(ListpFunction));
	ge->add(new Symbol("null?"), new BooleanUnary(NullpFunction));
	ge->add(new Symbol("primop?"), new BooleanUnary(PrimoppFunction));
	ge->add(new Symbol("closure?"), new BooleanUnary(ClosurepFunction));
	ge->add(new Symbol("print"), new UnaryFunction(PrintFunction));
	ge->add(new Symbol("lambda"), new LambdaFunction);
	ge->add(truesym, truesym);
	ge->add(new Symbol("nil"), emptyList());
}
\end{cprog}
\caption{Initialization of the Scheme Interpreter}\label{schemeinit}
\end{figure}


\chapter{The SASL interpreter}\label{sasl}

The SASL interpreter is largely constructed by removing features from the
Scheme interpreter, such as while loops and so on, and changing the 
implementation of the {\sf cons} function to add delayed evaluation.
Figure~\ref{saslhier} shows the class hierarchy for the classes added in
this chapter.

\setlength{\unitlength}{5mm}
\begin{figure}
\begin{picture}(25,10)(-4,-5)
\put(-3.5,0){\sf Expression}
\put(0,0.2){\line(1,0){1}}
\put(1,0){\sf Function}
\put(0,0.2){\line(1,-2){1}}
\put(1,-2){\sf Thunk}
\put(4,0.2){\line(1,-2){1}}
\put(5,-2){\sf UserFunction}
\put(9.5,-1.8){\line(1,0){1}}
\put(10.5,-2){\sf LazyFunction}
\put(4,0.2){\line(1,2){1}}
\put(5,2){\sf SaslConsFunction}
\put(4,0.2){\line(1,0){1}}
\put(5,0){\sf LambdaFunction}
\end{picture}
\caption{Class Hierarchy for expressions in the SASL interpreter}\label{saslhier}
\end{figure}

\subsection{Thunks}

Delayed evaluation is provided by adding a new expression type, called the
{\em thunk}.  Figure~\ref{thunk} shows the data structure used to represent
this type of value.  Every thunk maintains a boolean value indicating
whether the thunk has been evaluated yet, an expression (representing
either the unevaluated or evaluated expression, depending upon the state of
the boolean flag), and a context in which the expression is to be
evaluated.  Thunks print either as three dots, if they have not yet been
evaluated, or as the printed representation of their value, if they have.

\begin{figure}
\begin{cprog}
class Thunk : public Expression {
private:
	int evaluated;
	Expr value;
	Env context;
public:
	Thunk(Expression *, Environment *);

	virtual void free();
	virtual void print();
	virtual Expression * touch();
	virtual void eval(Expr &, Environment *, Environment *);

	virtual IntegerExpression * isInteger();
	virtual Symbol * isSymbol();
	virtual Function * isFunction();
	virtual ListNode * isList();
};

void Thunk::print()
{
	if (evaluated)
		value()->print();
	else
		printf("...");
}

Expression * Thunk::touch()
{
	// if we haven't already evaluated, do it now
	if (! evaluated) {
		evaluated = 1;
		Expression * start = value();
		if (start)
			start->eval(value, valueOps, context);
		}
	Expression * val = value();
	if (val)
		return val->touch();
	return val;
}
\end{cprog}
\caption{Definition of Thunks}\label{thunk}
\end{figure}

Here we finally see an overridden definition for the method {\em touch}.
You will recall that this method was defined in Chapter 1, and that all
other expressions merely return their value as the result of this
expression.  Thunks, on the other hand, will evaluate themselves if
touched, and then return their new evaluated result.  With the addition of
this feature many of the definitions we have presented in earlier chapters,
such as the definitions of {\sf car} and {\sf cdr}, hold equally well 
when given thunks as arguments.

Since thunks can represent lists, symbols, integers and so on, the
predicate methods {\sf isSymbol} and the like must be redefined as well.
If the thunk represents an evaluated value, these simply return the result
of testing that value (Figure~\ref{thunkpred}).

\begin{figure}
\begin{cprog}
void Thunk::eval(Expr & target, Environment * valusops, Environment * rho)
{
	touch();
	value()->eval(target, valusops, rho);
}

ListNode * Thunk::isList()
{
	// if its evaluated try it out
	if (evaluated) return value()->isList();

	// else it's not
	return 0;
}

Symbol * Thunk::isSymbol()
{
	if (evaluated) return value()->isSymbol();
	return 0;
}

Function * Thunk::isFunction()
{
	if (evaluated) return value()->isFunction();
	return 0;
}

IntegerExpression * Thunk::isInteger()
{
	if (evaluated) return value()->isInteger();
	return 0;
}
\end{cprog}
\caption{Thunk predicates}\label{thunkpred}
\end{figure}

\section{Lazy Cons}

The SASL cons function differs from the Scheme version in producing
a list node containing a pair of thunks, rather than a pair of values
(Figure~\ref{saslcons}).  Class {\sf SaslConsFunction} must now be a
subclass of {\sf Function} and not {\sf BinaryFunction}, because it must
grab its arguments before they are evaluated.  Thus it must itself check to
see that only two arguments are passed to the function.

\begin{figure}
\begin{cprog}
class SaslConsFunction : public Function {
public:
	virtual void apply(Expr & target, ListNode * args, Environment *);
};

void SaslConsFunction::apply(Expr & target, ListNode * args, Environment * rho)
{
	// check length
	if (args->length() != 2) {
		target = error("cons requires two arguments");
		return;
		}

	// make thunks for car and cdr
	target = new ListNode(new Thunk(args->at(0), rho), 
		new Thunk(args->at(1), rho));
}
\end{cprog}
\caption{The Sasl Lazy Cons function}\label{saslcons}
\end{figure}

\section{Lazy User Functions}

User defined functions must be provided with lazy evaluation semantics as
well.  This is accomplished by defining a new class {\sf LazyFunction}
(Figure~\ref{lazyfunction}).  Lazy functions act just like user functions
from previous chapters, only they do not evaluate their arguments.   Thus
the function body is evaluated by the method {\sf apply}, rather than
passing the evaluated arguments on to the method {\sf applyWithArgs}.
The lambda function from the previous chapter is modified to produce
an instance of {\sf LazyFunction}, rather than {\sf UserFunction}.

\begin{figure}
\begin{cprog}
class LazyFunction : public UserFunction {
public:
	LazyFunction(ListNode * n, Expression * b, Environment * c)
		: UserFunction(n, b, c) {}
	virtual void apply(Expr &, ListNode *, Environment *);
};

//	convert arguments into thunks
static ListNode * makeThunks(ListNode * args, Environment * rho)
{
	if ((! args) || (args->isNil()))
		return emptyList;
	Expression * newcar = new Thunk(args->head(), rho);
	return new ListNode(newcar, makeThunks(args->tail(), rho));
}

void LazyFunction::apply(Expr & target, ListNode * args, Environment * rho)
{
	// number of args should match definition
	ListNode * anames = argNames;
	if (anames->length() != args->length()) {
		error("argument length mismatch");
		return;
		}

	// convert arguments into thunks
	ListNode * newargs = makeThunks(args, rho);

	// make new environment
	Env newrho = new Environment(anames, newargs, context);

	// evaluate body in new environment
	if (body())
		body()->eval(target, valueOps, newrho);
	else
		target = 0;

	newrho = 0;
}
\end{cprog}
\caption{The implementation of lazy functions}\label{lazyfunction}
\end{figure}

\chapter{The CLU interpreter}

The CLU interpreter is created by introducing a new datatype, the cluster,
and three new types of functions.  Constructors create new instances of a
cluster, selectors access a portion of a cluster state, and modifiers change
a portion of a cluster state.  Figure~\ref{cluhier} shows the class
hierarchy for the classes added in this chapter.

\setlength{\unitlength}{5mm}
\begin{figure}
\begin{picture}(25,10)(-4,-5)
\put(-3.5,0){\sf Expression}
\put(0,0.2){\line(1,0){1}}
\put(1,0){\sf Function}
\put(0,0.2){\line(1,-2){1}}
\put(1,-2){\sf Cluster}
\put(4,0.2){\line(1,-2){1}}
\put(5,-2){\sf BinaryFunction}
\put(9.5,-1.8){\line(1,0){1}}
\put(10.5,-2){\sf Modifier}
\put(4,0.2){\line(1,2){1}}
\put(5,2){\sf UnaryFunction}
\put(9.5,2.2){\line(1,0){1}}
\put(10.5,2){\sf Selector}
\put(4,0.2){\line(1,0){1}}
\put(5,0){\sf Constructor}
\put(4,0.2){\line(1,4){1}}
\put(5,4){\sf ClusterDef}
\end{picture}
\caption{Class Hierarchy for the CLU interpreter}\label{cluhier}
\end{figure}

\section{Clusters}

A cluster simply encapsulates a series of names and values, hiding them
from normal examination.  The most natural way to do this is for a cluster
to maintain an environment (Figure~\ref{cluster}).  The predicate {\sf
isCluster} returns this environment value.

\begin{figure}
\begin{cprog}
class Cluster : public Expression {
private:
	Env data;
public:
	Cluster(ListNode * names, ListNode * values)
		{ data = new Environment(names, values, 0); }
	virtual void free()
		{ data = 0; }
	virtual void print()
		{ printf("<userval>"); }
	virtual Environment * isCluster()
		{ return data; }
};

class Constructor : public Function {
private:
	List names;
public:
	Constructor(ListNode * n);
	virtual void free();
	virtual void applyWithArgs(Expr &, ListNode *, Environment *);
};

void Constructor::applyWithArgs(Expr &target, ListNode *args, Environment *rho)
{
	ListNode * nmes = names;
	if (args->length() != nmes->length()) {
		target = error("wrong number of args passed to constructor");
		return;
		}
	target = new Cluster(nmes, args);
}
\end{cprog}
\caption{The definition of a cluster value}\label{cluster}
\end{figure}

To create a cluster requires a constructor function.  The constructor is
provided with a list of names of the elements in the internal
representation of the cluster, and simply insures that the arguments it is
provided with match in number of the names it maintains.

\section{Selectors and Modifiers}

To access or modify the elements of a constructor requires functions called
selectors or modifiers.  Each of these maintain as their state the name of
the field they are responsible for.  When invoked with a constructor, the
access or change their given field.

\begin{figure}
\begin{cprog}
class Selector : public UnaryFunction {
private:
	Expr fieldName;
public:
	Selector(Symbol * name);
	virtual void free();
	virtual void applyWithArgs(Expr &, ListNode *, Environment *);
};

void Selector::applyWithArgs(Expr & target, ListNode * args, Environment * rho)
{
	Environment * x = args->head()->isCluster();
	if (! x) {
		target = error("selector given non-cluster");
		return;
		}
	Symbol *s = fieldName()->isSymbol();
	if (!s)
		error("impossible case in selector, no symbol");
	target = x->lookup(s);
	if (! target())
		error("selector cannot find symbol:", s->chars());
}

class Modifier : public BinaryFunction {
private:
	Expr fieldName;
public:
	Modifier(Symbol * name);
	virtual void free();
	virtual void applyWithArgs(Expr &, ListNode *, Environment *);
};

void Modifier::applyWithArgs(Expr & target, ListNode * args, Environment * rho)
{
	Environment * x = args->head()->isCluster();
	if (! x) {
		target = error("selector given non-cluster");
		return;
		}

	// set the result to the value
	target = args->at(1);
	x->set(fieldName()->isSymbol(), target());
}
\end{cprog}
\caption{Selectors and Modifiers for clusters}
\end{figure}

\section{Defining clusters}

It thus remains only to give the (rather lengthy) definition of the
function that generates constructor information from the textual
description.  (We do not say generates clusters themselves, for that is the
responsibility of the constructor functions).  This function is shown in
Figure~\ref{clusterdef}.  It rips apart a cluster definition and does the
right things (need a better description here, but I don't have time to
write it now).  (Need to point out that cluster functions have an internal
and an external name, and these are put of different environments).
(I suppose an alternative would have been to introduce a new datatype for
two part names, which when evaluated would look up their second part in the
cluster provided by their first part).

\begin{figure}
\begin{cprog}
void ClusterDef::apply(Expr & target, ListNode * args, Environment * rho)
{
	Expr setprefix = new Symbol("set-");

	// must have at least name, rep and one def
	if (args->length() < 3) {
		target = error("cluster ill formed");
		return;
		}

	// get name
	Symbol * name = args->head()->isSymbol();
	args = args->tail();
	if (! name) {
		target = error("cluster missing name");
		return;
		}

	// now make the environment in which cluster will execute
	Environment * inEnv = new Environment(emptyList, emptyList, rho);

	// next part should be representation
	ListNode * rep = args->head()->isList();
	args = args->tail();
	if (! rep) {
		target = error("cluster missing rep");
		return;
		}
	Symbol *s = rep->at(0)->isSymbol();
	if (! (s && (*s == "rep"))) {
		target = error("cluster missing rep");
		return;
		}
	rep = rep->tail();

	// make the name into a constructor with the representation
	inEnv->add(name, new Constructor(rep));
\end{cprog}
\caption{The cluster recognition function}\label{clusterdef}
\end{figure}
\begin{figure}
\begin{cprog}
	// now run dow the rep list, making accessor functions
	while (! rep->isNil()) {
		s = rep->head()->isSymbol();
		if (! s) {
			target = error("ill formed rep in cluster");
			return;
			}
		inEnv->add(s, new Selector(s));
		catset(inEnv, setprefix()->isSymbol(), "",
			s, new Modifier(s));
		rep = rep->tail();
		}
	
	// remainder should be define commands
	while (! args->isNil()) {
		ListNode * body = args->head()->isList();
		if (! body) {
			target = error("ill formed cluster");
			return;
			}
		s = body->at(0)->isSymbol();
		if (! (s && (*s == "define"))) {
			target = error("missing define in cluster");
			return;
			}
		s = body->at(1)->isSymbol();
		if (! s) {
			target = error("missing name in define");
			return;
			}

		// evaluate body to define new function
		Expr temp;
		body->eval(temp, commands, inEnv);
		// make outside form
		catset(rho, name, "$", s, inEnv->lookup(s));
		temp = 0;

		// get next function
		args = args->tail();
		}

	// what do we return?
	target = 0;
	setprefix = 0;
}

static void catset(Environment * rho, Symbol * left, char * mid, 
		Symbol * right, Expression * val)
{	char buffer[120];

	// catenate the two symbols
	strcpy(buffer, left->chars());
	strcat(buffer, mid);
	strcat(buffer, right->chars());

	// now put the new value into rho
	rho->add(new Symbol(buffer), val);
}
\end{cprog}
\caption{The cluster recognition function (continued)}
\end{figure}

\chapter{The Smalltalk interpreter}

As with chapter 3, with the Smalltalk interpreter I have also made a number
of changes.  These include the following:

\begin{itemize}
\item
I have changed the syntax for message passing.
The first argument in a message
passing expression is an object, which is defined (for implementation
purposes) as a type of function.  The second argument must be the message
selector, a symbol.  This change is not only produces a syntax that is
slightly more Smalltalk-like, but it more closely reinforces the critical
object-oriented idea that the interpretation of a message depends upon the
receiver for that message.
\item
Integers are objects, and respond to messages.  The most obvious effect of
this is to restore infix syntax for arithmetic operations, since (3 + 4) is
interpreted (Smalltalk-like) as the message ``+'' being passed to the
object 3 with argument 4.
\item
The initial environment is very spare.  There are only the two classes
{\sf Object} and {\sf Integer}, which respond to the messages 
{\sf subclass}, {\sf method} and {\sf new}, and integer instances that respond
to arithmetic messages.
\item
The {\sf if} command is a message sent to integers (0 for false and
non-zero for true).  This is also more Smalltalk-like.  The following
expression sets {\sf z} to the minimum of {\sf x} and {\sf y}.
\begin{center}
{\sf ((x $<$ y) if (set z x) (set z y))}
\end{center}
\item
The only non-message statements are the assignment statement {\sf set} and
the {\sf begin} statement.  (Note - there is no loop.  I couldn't think of
a good way to do this within the syntax given using message passing (no
blocks!) but I don't think this will be too great a problem; recursion can
be used in most cases where looping is used currently).
\end{itemize}

A class hierarchy for the classes added in this chapter is shown in
Figure~\ref{smallhier}.

\setlength{\unitlength}{5mm}
\begin{figure}
\begin{picture}(25,10)(-4,-5)
\put(-3.5,0){\sf Expression}
\put(0,0.2){\line(1,0){1}}
\put(1,0){\sf Function}
\put(0,0.2){\line(1,4){1}}
\put(1,4){\sf Symbol}
\put(3,4.2){\line(1,0){1}}
\put(4,4){\sf SmalltalkSymbol}
\put(4,0.2){\line(1,-2){1}}
\put(5,-2){\sf UserFunction}
\put(9,-1.8){\line(1,0){1}}
\put(10,-2){\sf method}
\put(13,-1.8){\line(1,0){1}}
\put(14,-2){\sf SubclassMethod}
\put(13,-1.8){\line(1,1){1}}
\put(14,-1){\sf NewMethod}
\put(13,-1.8){\line(1,2){1}}
\put(14,0){\sf IntegerBinaryMethod}
\put(13,-1.8){\line(1,-1){1}}
\put(14,-3){\sf IfMethod}
\put(13,-1.8){\line(1,-2){1}}
\put(14,-4){\sf MethodMethod}
\put(4,0.2){\line(1,2){1}}
\put(5,2){\sf Object}
\put(7,2.2){\line(1,0){1}}
\put(8,2){\sf IntegerObject}
\end{picture}
\caption{Expression class hierarchy for Smalltalk interpreter}\label{smallhier}
\end{figure}

\section{Objects and Methods}

An object is an encapsulation of behavior and state.  That is, an object
maintains, like a cluster, certain state information accessible only
within the object.  Similarly objects maintain a collection of functions,
called {\em methods}, that can be invoked only via message passing.
Internally, both these are represented by environments (Figure~\ref{obj}).
The methods environment contains a collection of functions, and the data
environment contains a collection of internal variables.  Objects are
declared as a subclass of {\sf function} so that normal function syntax can be
used for message passing.  That is, a message is written as

\begin{center}
{\sf (object message arguments)}
\end{center}

\begin{figure}
\begin{cprog}
class Object : public Function {
private:
	Env methods;
	Env data;
	friend class SubclassMethod;
public:
	Object(Environment * m, Environment * d);

	virtual void print();
	virtual void free();
	virtual void apply(Expr &, ListNode *, Environment *);

	// methods used by classes to create new instances
	// note these are invoked only on classes, not simple instances
	ListNode * getNames();
	Environment * getMethods();
};

class Method : public UserFunction {
public:
	Method(ListNode *anames, Expression * bod) ;

	virtual void doMethod(Expr&, Object*, ListNode*, 
		Environment*, Environment*);

	virtual Method * isMethod();
};
\end{cprog}
\caption{Classes for Object and Method}\label{obj}
\end{figure}

Methods are similar to conventional functions (and are thus subclasses of
{\sf UserFunction}) in that they have an argument list and body.  Unlike
conventional functions they have a receiver (which must always be an object)
and the environment in which the method was created, as well as the
environment in which the method is invoked.  Thus methods define a new
message {\sf doMethod} that takes these additional arguments.

A subtle point to note is that the creation environment in normal functions
is captured when the function is defined.  For objects this environment
cannot be defined when the methods are created, but must wait until a new
instance is created.  Our implementation waits even longer, and passes it 
as part of the message passing protocol.

The mechanism of message passing is defined by the function {\sf apply} in
class {\sf Object} (Figure~\ref{apply}).  Messages require a symbol for the
first argument, which must match a method for the object.  This method is
then invoked.  Similarly Figure~\ref{apply} shows the execution of normal
methods (that is, those methods other than the ones provided by the
system).  The execution context is set for
the method, and the receiver is added as an implicit first argument, 
called {\sf self} in every method.  The method is then invoked as if it
were a conventional function.

\begin{figure}
\begin{cprog}
void Object::apply(Expr & target, ListNode * args, Environment * rho)
{
	// need at least a message
	if (args->length() < 1) {
		target = error("ill formed message expression");
		return;
		}

	Symbol * message = args->head()->isSymbol();
	if (! message) {
		target = error("object needs message");
		return;
		}

	// now see if message is a method
	Environment * meths = methods;
	Expression * methexpr = meths->lookup(message);
	Method * meth = 0;
	if (methexpr) meth = methexpr->isMethod();
	if (! meth) {
		target = error("unrecognized method name: ", message->chars());
		return;
		}

	// now just execute the method (take off message from arg list)
	meth->doMethod(target, this, args->tail(), data, rho);
}

void Method::doMethod(Expr& target, Object* self, ListNode* args, 
	Environment *ctx, Environment *rho)
{
	// change the execution context
	context = ctx;

	// put self on the front of the argument list
	List newargs = new ListNode(self, args);

	// and execute the function
	apply(target, newargs, rho);

	// clean up arg list
	newargs = 0;
}
\end{cprog}
\caption{Implementation of Message Passing}\label{apply}
\end{figure}

\section{Classes}

Classes are simply objects.  As such, they respond to certain messages.
In our Smalltalk interpreter there are initially two classes, {\sf Object}
and {\sf Integer}.  
The class {\sf Object} is a superclass of {\sf Integer}, and is typically
the superclass of most user defined classes as well.
There are initially three messages that classes respond to:

\begin{itemize}
\item
{\sf subclass}.  This message is used to create new classes, as subclasses
of existing classes.  Any arguments provided are treated as the names of
instance variables (local state) to be generated when instances of the new
classes are created.  The new class is returned as an object, and is
usually immediately assigned to a global variable.
The syntax for new classes is thus similar to the following:
\begin{center}
{\sf (set Foo (Object subclass x y z))}
\end{center}
which creates a new class with three instance variables, and assigns this
class to the variable {\sf Foo}.  Subclasses can also access instance
variables defined in classes.

It is legal to subclass from class {\sf Integer}, although the results are not
useful for any purpose.

\item
{\sf new}.  This message, which takes no arguments, is used to create a new
instance of the receiver class.  The new instance is returned as the result
of the method, as in the following:
\begin{center}
{\sf (set newfoo (Foo new))}
\end{center}
Although the class {\sf Integer} responds to the message {\sf new}, no
useful value is returned.  (Real Smalltalk has something called 
{\em metaclasses} that can be used to prevent certain classes from
responding to all messages.  Our Smalltalk doesn't).
\item
{\sf method}.  This message is used to define a new method for a class.
Following the keyword {\sf method} the syantx is the same as a normal
function definition.  Within a method the pseudo-variable {\sf self} can be
used to represent the receiver for the method.
\begin{center}
{\sf (Integer method square () (self * self))}
\end{center}
\end{itemize}

Classes are represented in the same format as other objects.  They act as
if they held two instance variables; {\sf names}, which contains a list of
instance variable names for the class, and {\sf methods}, which contains
the table of method definitions for the class.  Note that these are held in
the data area for the class.  (A picture might help here...).

The implementation of the method {\sf subclass} is shown in
Figure~\ref{subclass}.  The instance variables for the parent class is
obtained, and the new instance variables for the class added to them.
Inheritance is implemented by creating a new empty method table, but having
it point to the method table for the parent class.  Thus a search of the
method table for the newly created class will automatically search the
parent class if no overriding method is found.  These two values are
inserted as data values in the new class object.  The methods a class
responds to will be exactly the same as those of the parent class (thus all
classes respond to the same messages).

\begin{figure}
\begin{cprog}
void SubclassMethod::doMethod(Expr& target, Object* self, ListNode *args,
	Environment *ctx, Environment *rho)
{

	// the argument list is added to the list of variables
	ListNode * vars = self->getNames();
	while (! args->isNil()) {
		vars = new ListNode(args->head(), vars);
		args = args->tail();
		}

	// the method table is empty, but points to inherited method table
	Environment * newmeth = new Environment(emptyList, emptyList,
			self->getMethods());

	// make the new data area
	Environment * newEnv = new Environment(emptyList, emptyList, rho);
	newEnv->add(new Symbol("names"), vars);
	newEnv->add(new Symbol("methods"), newmeth);

	// now make the new object
	Environment * meths = self->methods;
	target = new Object(meths, newEnv);
}

ListNode * Object::getNames()
{
	Environment * datavals = data;
	Expression * x = datavals->lookup(new Symbol("names"));
	if ((! x) || (! x->isList())) {
		error("impossible case in Object::getNames");
		return 0;
		}
	return x->isList();
}
\end{cprog}
\caption{Implementation of the {\sf subclass} method}\label{subclass}
\end{figure}

The implementation of the method {\sf new}, shown in
Figure~\ref{newmethod}, gets the list of instance variables associated
with the class.  A new environment is then created that assigns an empty
value to each variable.  Using the method table stored in the data area for
the class object a new object is then created.

\begin{figure}
\begin{cprog}
void NewMethod::doMethod(Expr& target, Object* self, ListNode *args,
	Environment *ctx, Environment *rho)
{
	// get the list of instance names
	ListNode * names = self->getNames();

	// cdr down the list, making a list of values (initially zero)
	ListNode * values = emptyList;
	for (ListNode *p = names; ! p->isNil(); p = p->tail())
		values = new ListNode(new IntegerExpression(0), values);

	// make the new environment for the names
	Environment * newenv = new Environment(names, values, rho);

	// make the new object
	target = new Object(self->getMethods(), newenv);
}
\end{cprog}
\caption{The method {\sf new}}\label{newmethod}
\end{figure}

\begin{figure}
\begin{cprog}
void MethodMethod::doMethod(Expr& target, Object* self, ListNode *args,
	Environment *ctx, Environment *rho)
{
	if (args->length() != 3) {
		target = error("method definition requires three arguments");
		return;
		}
	Symbol * name = args->at(0)->isSymbol();
	if (! name) {
		target = error("method definition missing name");
		return;
		}

	ListNode * argNames = args->at(1)->isList();
	if (! argNames) {
		target = error("method definition missing arg names");
		return;
		}
	// put self on front of arg names
	argNames = new ListNode(new Symbol("self"), argNames);

	// get the method table for the given class
	Environment * methTable = self->getMethods();

	// put method in place
	methTable->add(name, new Method(argNames, args->at(2)));

	// yield as value the name of the function
	target = name;
}
\end{cprog}
\caption{The method {\sf method}}\label{methodmethod}
\end{figure}

The method used to respond to the {\sf method} command is shown in
Figure~\ref{methodmethod}.  This is very similar to the function used to break
apart the {\sf define} command in Chapter 1.   The only significant
difference includes the addition of the receiver {\sf self} as an implicit
first parameter in the argument list, and the fact that the function is
placed in a method table, rather than in the global environment.

\section{Symbols and Integers}

Symbols in Smalltalk have no property other than they evaluate to
themselves, and are guaranteed unique.
They are easily implemented by subclassing the existing class {\sf Symbol}
(Figure~\ref{smsymbol}), and modifying the reader/parser to recognize the
tokens.  (Unlike symbols in real Smalltalk, our symbols
are not objects and will not respond to any messages).

\begin{figure}
\begin{cprog}
class SmalltalkSymbol : public Symbol {
public:
	virtual void eval(Expr & target, Environment *, Environment *)
		{ target = this; }
};

static Env IntegerMethods;

class IntegerObject : public Object {
private:
	Expr value;
public:
	IntegerObject(int v) : Object(IntegerMethods, 0) 
		{ value = new IntegerExpression(v); }

	virtual void print()
		{ if (value()) value()->print(); }

	virtual void free()
		{ value = 0; }

	virtual IntegerExpression * isInteger()
		{ if (value()) return value()->isInteger(); return 0; }
};
\end{cprog}
\caption{Symbols and Integers in Smalltalk}\label{smsymbol}
\end{figure}

Integers are also redefined as objects, and a built-in method {\sf
IntegerBinaryMethod}, similarly to {\sf IntegerBinaryFunction}, is created
to simplify the arithmetic methods.

Control flow is implemented as a message to integers.  (In real Smalltalk
control flow is implemented as messages, but to different objects).
If the receiver is zero the first argument to the if method is returned, 
otherwise the second argument is returned.

\begin{figure}
\begin{cprog}
void IfMethod::doMethod(Expr & target, Object * self, 
	ListNode * args, Environment * ctx, Environment * rho)
{
	if (args->length() != 2) {
		target = error("wrong number of args for if");
		return;
		}
	IntegerExpression * cond = self->isInteger();
	if (! cond) {
		target = error("impossible!", "no cond in if");
		return;
		}
	if (cond->val())
		args->at(0)->eval(target, valueOps, rho);
	else
		args->at(1)->eval(target, valueOps, rho);
}
\end{cprog}
\caption{Implementation of the if method}
\end{figure}

\section{Smalltalk reader}

The Smalltalk reader subclasses the reader class so as to recognize
integers and symbols (Figure~\ref{smalltalkreader}).

\begin{figure}
\begin{cprog}
Expression * SmalltalkReader::readExpression()
{
	// see if it's an integer
	if (isdigit(*p))
		return new IntegerObject(readInteger());

	// might be a signed integer
	if ((*p == '-') && isdigit(*(p+1))) {
		p++;
		return new IntegerObject(- readInteger());
		}

	// or it might be a symbol
	if (*p == '#') {
		char token[80], *q;

		for (q = token; ! isSeparator(*p); )
			*q++ = *p++;
		*q = '\0';
		return new SmalltalkSymbol(token);
		}

	// anything else, do as before
	return Reader::readExpression();
}
\end{cprog}
\caption{The Smalltalk reader}\label{smalltalkreader}
\end{figure}

\section{The big bang}

To initialize the interpreter we must create the objects {\sf Object} and
{\sf Integer}.  (Need more explanation here, but I'll just give the code
for now).

\begin{figure}
\begin{cprog}
initialize()
{
	// initialize global variables
	reader = new SmalltalkReader;

	// the only commands are the assignment command  and begin
	Environment * vo = valueOps;
	vo->add(new Symbol("set"), new SetStatement);
	vo->add(new Symbol("begin"), new BeginStatement);

	// initialize the global environment
	Environment * ge = globalEnvironment;

	// first create the object ``Object''
	Environment* objMethods = new Environment(emptyList, emptyList, 0);
	Environment* objClassMethods = new Environment(emptyList, emptyList,
				objMethods);
	objClassMethods->add(new Symbol("new"), new NewMethod);
	objClassMethods->add(new Symbol("subclass"), new SubclassMethod);
	objClassMethods->add(new Symbol("method"), new MethodMethod);
	Environment * objData = new Environment(emptyList, emptyList, 0);
	objData->add(new Symbol("names"), emptyList());
	objData->add(new Symbol("methods"), objMethods);
	ge->add(new Symbol("Object"), 
			new Object(objClassMethods, objData));

	// now make the integer methods
	IntegerMethods = new Environment(emptyList, emptyList, objMethods);
	Environment * im = IntegerMethods;
	// the integer methods are just as before
	im->add(new Symbol("+"), new IntegerBinaryMethod(PlusFunction));
	im->add(new Symbol("-"), new IntegerBinaryMethod(MinusFunction));
	im->add(new Symbol("*"), new IntegerBinaryMethod(TimesFunction));
	im->add(new Symbol("/"), new IntegerBinaryMethod(DivideFunction));
	im->add(new Symbol("="), new IntegerBinaryMethod(IntEqualFunction));
	im->add(new Symbol("<"), new IntegerBinaryMethod(LessThanFunction));
	im->add(new Symbol(">"), new IntegerBinaryMethod(GreaterThanFunction));
	im->add(new Symbol("if"), new IfMethod);
	ge->add(new Symbol("Integer"),
			new Object(objClassMethods, objData));
}
\end{cprog}
\caption{Initializing the Smalltalk interpreter}
\end{figure}

\chapter{The Prolog interpreter}

As with chapters 3 and 7, I have in this chapter taken great liberties with
the syntax used by Kamin in his interpreter.  However, unlike chapters 3 and 7,
where my intent was to make the interpreters closer in spirit to the
original language, my intent here is to simplify the interpreter.
Specifically, I wanted to build on the base interpreter, just as we have done
for all other languages.  I am able to do this by adopting {\em continuations}
as the fundamental basis for my implementation, and by basing the code on
slightly different primitives.

The language used by this interpreter has the following characteristics:
\begin{itemize}
\item
As in real prolog, the only basic objects are symbols.
I've even tossed out integers, just to simplify things.
Symbols have no meaning other than their uniqueness.
Those symbols beginning with lower case letters are atomic, while those
beginning with upper case letters are variables.
\item
There are two basic statement types, the {\sf define} statement we
have seen all through the interpreters, and a new statement called 
{\sf query}.   The later is used to form questions.
\item
The bodies of functions or queries can be composed of four types of
relations:
\begin{itemize}
\item
{\sf (print x)} which if x is defined prints the value of x and is successful,
and if x is not defined is not successful.
\item
{\sf (:=: x y)} which attempts to unify x and y, which can be either 
variables or symbols.  The order of arguments is unimportant.
\item
{\sf (and $rel_1$ $rel_2$ ... )} which can take any number of 
relational arguments and is successful if all the relations are successful.  
Relations are tried in order.
\item
{\sf (or $rel_1$ $rel_2$ ... )} which can take any number of relational 
arguments and is successful if one one of the relations is successful.  
They are tried in order.
\end{itemize}
\end{itemize}

For example, suppose {\sf sam} is the father of {\sf alice}, and {\sf alice} 
is the mother of {\sf sally}.  We might encode this in a parent database 
as follows:

\begin{cprog}
-> (define parent (X Y)
	(or
		(and (:=: X alice) (:=: Y sally))
		(and (:=: X sam) (:=: Y alice)) 
	))
\end{cprog}

The query statement can then be used to ask queries of the database.
For example, we can find out who is the parent of {\sf alice} as follows:

\begin{cprog}
-> (query (and (parent X alice) (print X)))
sam
ok
\end{cprog}

Or we can find the child of {\sf alice} with the following:

\begin{cprog}
-> (query (and (parent alice X) (print X)))
sally
ok
\end{cprog}

If we ask a question that does not have an answer, the response not-ok is
printed.

\begin{cprog}
-> (query (and (parent fred X) (print X)))
not ok
\end{cprog}

Prolog style rules can be introduced using the same form we have been using
for functions.

\begin{cprog}
-> (define grandparent (X Y)
	(and (parent X Z) (parent Z Y)))
-> (query (and (grandparent A B) (print A) (print B)))
sam
sally
ok
\end{cprog}

There is no built-in way to force a relation to cycle through all
alternatives.  However, this is easily accomplished by making a relation
that will always fail, for example trying to unify apples with oranges:

\begin{cprog}
-> (define fail () (:=: apples oranges))
\end{cprog}

We can then use this to print out all the parents in our database.
Notice that not-ok is printed, since we eventually fail.

\begin{cprog}
-> (query (and (parent X Y) (print X) (fail)))
sam
alice
not ok
\end{cprog}

Note - although it might appear the use of and's and or's is more powerful
than writing rules in horn clauses, in fact they are identical; although
horn clauses will often require the introduction of unnecessary names.
I myself find this formulation more natural, although I'm not exactly
unbiased.

Unification of two unknown symbols works as expected.  If any symbol
subsequently becomes defined, the other is defined as well.
\begin{cprog}
-> (define same (X Y) (:=: X Y))
-> (query (and (same A B) (:=: A sally) (print B)))
sally
ok
\end{cprog}

I will divide the discussion of the implementation into three parts.
These are unification, symbol management, and backtracking.

\section{Unification}

Unification is the basis for logic programming.  Using unification, unbound
variables can be bound together.  As we saw in the last example, this is
more than simple assignment.  If two unknown variables are unified together
and subsequently one is bound, the other should be bound also.
Unification also differs
from assignment in that it can be ``undone'' during the process of backtracking.

Unification is most easily implemented by introducing a level of
indirection.  Prolog values will be represented by a new type of
expression, called {\sf PrologValue} (Figure~\ref{prologvalue}).
Instances of this class maintain a data value, which is either undefined
(that is, null), a symbol, or another prolog value.
The prolog reader is modified so as to return a prolog value were formerly a
symbol was returned.  (Also the reader will no longer recognize integers,
which are not used in our simplified interpreter).

A prolog value that contains a symbol is used to represent the prolog
symbol of the same name.
A prolog value that contains an empty data value represents a currently
unbound value.  Finally a prolog value that points to another prolog value
represents the unification of the first value with the second.
Whenever we need the value of a prolog symbol, we first run down the chain
of indirections to get to the bottom of the sequence.
(This is done automatically by the overridden method {\sf isSymbol}, which
will yield the symbol value behind arbitrary levels of indirection if a
prolog value represents a symbol.)

\begin{figure}
\begin{cprog}
class PrologValue : public Expression {
private:
	Expr data;

public:
	PrologValue(Expression * d) { data = d; }

	virtual void free() { data = 0; }
	virtual void print();
	virtual void eval(Expr &, Environment *, Environment *);
	virtual Symbol * isSymbol();
	virtual PrologValue * isPrologValue() { return this; }

	int isUndefined() 			{ return data() == 0; }
	void setUndefined() 			{ data = 0; }
	PrologValue * indirectPtr();
	void setIndirect(PrologValue *v) 	{ data = v; }
};
\end{cprog}
\caption{The class declaration for prolog values}\label{prologvalue}
\end{figure}

The unification algorithm is shown in Figure~\ref{unify}.  For reasons we
will return to when we discuss backtracking, the algorithm takes three
arguments.  The first is a reference to a pointer to a prolog value.
If the unification process changes the value of either the the two other
arguments, the pointer in the first argument is set to the altered value.

\begin{figure}
\begin{cprog}
static int unify(PrologValue *& c, PrologValue * a, PrologValue * b)
{

	// if either one is undefined, set it to the other
	if (a->isUndefined()) {
		c = a;
		a->setIndirect(b);
		return 1;
		}
	if (b->isUndefined()) {
		c = b;
		b->setIndirect(a);
		return 1;
		}

	// if either one are indirect, run down chain
	PrologValue * indirval;
	indirval = a->indirectPtr();
	if (indirval)
		return unify(c, indirval, b);
	indirval = b->indirectPtr();
	if (indirval)
		return unify(c, a, indirval);

	// both must now be symbolic, work if the same
	c = 0;
	Symbol * as = a->isSymbol();
	Symbol * bs = b->isSymbol();
	if ((! as) || (! bs)) 
		error("impossible", "unification of non-symbols");
	else if (strcmp(as->chars(), bs->chars()) == 0)
		return 1;
	return 0;
}
\end{cprog}
\caption{The Unification process}\label{unify}
\end{figure}

The unification process divides naturally into three parts.  If either
argument is undefined, it is changed so as to point to the other arguments.
This is true regardless of the state of the other argument.  This is how
two undefined variables can be unified - the first is set to point to the
second.   If the second is subsequently changed, the first will still
indirectly point to the new value.  Suppose it is, however, the first that
is subsequently changed?  In that case the next portion of the unification
algorithm is entered.  If both arguments are defined and either one is an
indirection, then we simply try to unify the next level down in the pointer
chains.  (Note: Lots of pictures would make this clearer, but I don't have
time right now..).  If neither argument is undefined nor an indirection,
they both must be symbols.  In that case, unification is successful if and
only if they have the same textual representation.

\section{Symbol Management}

The only significant problem here is that symbolic constants must evaluate to 
themselves and that symbolic variables can be introduced without
declaration.  We see the latter in sequences such as:

\begin{cprog}
-> (define grandparent (X Y)
	(and (parent X Z) (parent Z Y)))
-> (query (and (grandparent A B) (print A) (print B)))
sam
sally
ok
\end{cprog}

Here the variable Z suddenly appears without prior use.  The solution to
both of these problems is found in the code used to respond to the {\sf
eval} request for a Prolog value.  This code is shown in
Figure~\ref{prologeval}.  The virtual method {\sf isSymbol} runs down any
indirection links, returning the symbol data value if the last value in a
chain of indirections represents a symbolic constant.
If a symbol is found, we first look to see if the symbol is bound in the
current environment.  If so we simply return its binding\footnote{Checking
for bindings before checking the first letter allows rules to begin with
lower case letters, which seems to be more natural to most programmers.}.  
If not, if the
symbol begins with a lower case letter it evaluates to itself, and so we
simply return it.  If it is not a symbolic constant, than it is a new
symbolic variable, and we add a binding to the current environment to
indicate that the value is so-far undefined.  Thus new symbols are added to
the current environment as they are encountered, instead of generating
error messages as they did in previous interpreters.

\begin{figure}
\begin{cprog}
Symbol * PrologValue::isSymbol()
{
	PrologValue * iptr = indirectPtr();
	if (iptr)
		return iptr->isSymbol();
	if (! isUndefined())
		return data()->isSymbol();
	return 0;
}

void PrologValue::eval(Expr&target, Environment*valueOps, Environment*rho)
{
	Symbol * s = isSymbol();
	if (s) {
		char * p = s->chars();
		Expression * r = rho->lookup(s);
		if (r) {
			target = r;
			return;
			}
		// symbol is not known
		// if lower case, eval to itself
		if ((*p >= 'a') && (*p <= 'z')) {
			// symbols eval to themselves
			target = this;
			return;
			}
		// else make a new symbol
		target = new PrologValue(0);
		rho->add(s, target());
		return;
		}
	target = this;
	return;
}
\end{cprog}
\caption{Evaluation of a prolog symbol}\label{prologeval}
\end{figure}

\section{Backtracking}

The seem to be two general approaches to implementing logic programming
languages.  The technique used by most modern prolog systems is called the
WAM, or Warren Abstract Machine.  The WAM performs backtracking by not
popping the activation frame stack when a procedure is terminated, and
saving enough information to restart the procedure in a record called the
``choice point''.  Since in our interpreters calling a function is performed
by recursively calling evaluation routines inside the interpreter, the
activation stack for the users program is held in part in the activation
stack for the interpreter itself.  Thus it is difficult for us to
manipulate the activation record stack directly.
The alternative technique, which is actually historically older, is to
build up an unevaluated expression that represents what it is you want to 
do next before
you ever start execution.  This is called a continuation, and we were
introduced to this idea in the chapter on Scheme.
When we are faced with a choice, we can then try one alternative and the
continuation, and if that doesn't work try the next.

In general continuations are simply arbitrary expressions representing
``what to do next''.  In our case they will always return a boolean value,
indicating whether they are to be considered successful or not.
We will sometimes refer to the continuation as the ``future'', since it
represents the calculation we want to perform in the future.

In order to illustrate how backtracking can be implemented using
continuation, let us consider the following invocation of our family
database:

\begin{cprog}
(query (and (grandparent sam A) (print A)))
\end{cprog}

There are two important points to note.  The first is that the general
approach will be a two step process, construct the future that represents
the calculation we want to do, then do it.  The second point is that the
details are exceedingly messy; you should be eternally grateful that it is
the computer that is performing this task, and not you.

To begin, the continuation that represents what it is we want to do after
evaluating the query is the null continuation, an expression that merely
returns true.
In order to try to keep track of the multiple levels of evaluation, let us
write this as follows:

\begin{cprog}
(and (grandparent sam A) (print A)) [ true ]
\end{cprog}

This says that we want to evaluate the {\sf and} relation, and then do the
calculation given by the bracketed expression.

Consider now the meaning of {\sf and}.  The and expression should evaluate
the first relation, and if successful evaluate the second, and finally if
that is successful evaluate the future given to the original expression.
What then is the ``future'' of the first relation?  It is simply the second
relation and the original future.  That is, the calculation we want to
perform if the first relation is successful is simply the following:

\begin{cprog}
(print A) [ true ]
\end{cprog}

We can wrap this in a bracket in order to make a continuation in
our form out of it.  
Using {\em this} as the future for the first relation gives us the
following:

\begin{cprog}
(grandparent sam A) [ (print A) [ true ] ]
\end{cprog}

We are in effect turning the calculation inside out.  We have replaced the
{\sf and} conjunction with a list of expressions to evaluate in the future.

The invocation of the {\sf grandparent} relation causes the expression to
be replaced by the function definition, with the arguments suitably bound
to the parameters.   That is, the effect is the same as:\footnote{We will
use textual replacement of the parameters by the arguments in our example,
although in practice the effect is achieved via a level of indirection
provided by environments, as in all the interpreters we have studied.}

\begin{cprog}
(and (parent sam Z) (parent Z A)) [ (print A) [ true ] ]
\end{cprog}

We have already analyzed the meaning of the {\sf and} relation.  The future
we want to provide for the first relation is the expression yielded by:

\begin{cprog}
(parent Z A) [ (print A) [ true ] ]
\end{cprog}

As before, we can expand the invocation of the {\sf parent} relation by
replacing it by its definition, making suitable transformations of the
argument values.

\begin{cprog}
(or
	(and (:=: Z alice) (:=: A sally))
	(and (:=: Z sam) (:=: A alice)) )
		[ (print A) [ true ] ]
\end{cprog}

The {\sf or} relation should try each alternative in turn, passing it as
the future the continuation passed to the or.  If any is successful we
should return success, otherwise the or should fail.
Thus we can distribute the future to each clause of the or, and rewrite it
as follows:\footnote{The fact that we are replacing or by a conditional 
may seem odd,
but the more important point is that we have moved the evaluation of the
future down to each of the arguments to the or expression.}

\begin{cprog}
if (and (:=: Z alice) (:=: A sally))
		[ (print A) [ true ] ]
then return true
else if (and (:=: Z sam) (:=: A alice)) )
		[ (print A) [ true ] ]
then return true
else return false
\end{cprog}

If we perform the already-defined transformations on the {\sf and} 
relations we obtain the following:

\begin{cprog}
if (:=: Z alice) [ (:=: A sally) [ (print A) [ true ] ] ]
then return true
else if (:=: Z sam) [ (:=: A alice) [ (print A) [ true ] ] ]
then return true
else return false
\end{cprog}

Recall that this was all performed just to construct the continuation 
for the first clause in an earlier expression.
Thus the expression we are now working on is as follows:

\begin{cprog}
(parent sam Z) [
	if (:=: Z alice) [ (:=: A sally) [ (print A) [ true ] ] ]
	then return true
	else if (:=: Z sam) [ (:=: A alice) [ (print A) [ true ] ] ]
	then return true
	else return false ]
\end{cprog}

We before, we can expand the call on {\sf parent} by its
definition:\footnote{I warned you about the messy details!}

\begin{cprog}
(or
	(and (:=: sam alice) (:=: Z sally))
	(and (:=: sam sam) (:=: Z alice)))
[ if (:=: Z alice) [ (:=: A sally) [ (print A) [ true ] ] ]
	then return true
	else if (:=: Z sam) [ (:=: A alice) [ (print A) [ true ] ] ]
	then return true
	else return false ]
\end{cprog}

Once again distributing the future along each argument of the or expression
yields:

\begin{cprog}
if (and (:=: sam alice) (:=: Z sally))
	[ if (:=: Z alice) [ (:=: A sally) [ (print A) [ true ] ] ]
	then return true
	else if (:=: Z sam) [ (:=: A alice) [ (print A) [ true ] ] ]
	then return true
	else return false ]
then return true
else if (and (:=: sam sam) (:=: Z alice))
	[ if (:=: Z alice) [ (:=: A sally) [ (print A) [ true ] ] ]
	then return true
	else if (:=: Z sam) [ (:=: A alice) [ (print A) [ true ] ] ]
	then return true
	else return false ]
else return false
\end{cprog}

Performing yet one more time the transformations on the {\sf and} relations
yields:

\begin{cprog}
if (:=: sam alice) [ (:=: Z sally) 
	[ if (:=: Z alice) [ (:=: A sally) [ (print A) [ true ] ] ]
	then return true
	else if (:=: Z sam) [ (:=: A alice) [ (print A) [ true ] ] ]
	then return true
	else return false ] ]
then return true 
else if (:=: sam sam) [ (:=: Z alice)
	[ if (:=: Z alice) [ (:=: A sally) [ (print A) [ true ] ] ]
	then return true
	else if (:=: Z sam) [ (:=: A alice) [ (print A) [ true ] ] ]
	then return true
	else return false ]
then return true
else return false
\end{cprog}

This is the final continuation that is constructed by the query expression.
The most important feature of this expression is that it can be evaluated
in a forward fashion, without backtracking.
Having generated it, the next step is execution.  Contrast this with the
description we provided earlier.  First an attempt is made to unify the
symbols {\sf sam} and {\sf alice}.  This fails, and thus the continuation 
for the first conditional is ignored.  Next an attempt is made to unify 
the symbols {\sf sam} and {\sf sam}.  This is successful, and thus we 
evaluate the continuation to the next expression.  The continuation unifies 
{\sf Z} and {\sf alice}, binding the left-hand variable to the right-hand 
symbol.  The continuation for that expression then trys to unify {\sf Z} and 
{\sf alice}, which is successful.  Thus variable {\sf A} is bound to 
{\sf sally}, and is printed.\footnote{Observant readers will have noted
that some of the conditionals could have been evaluated during the
construction of the continuation.  This is true, and is an important
optimization in real systems.}

Having described the general approach our interpreter will follow, we will
now go on to provide the specific details.

Our continuations are built around a new datatype, which we will call the
{\sf Continuation}.
A continuation should be thought of as an unevaluated boolean expression.
The continuation performs some action, which may
or may not succeed.   The success of the action is indicated by the boolean
value returned.  
The class Continuation is shown in Figure~\ref{relation}.  The routine used to
invoke a relation is the virtual method {\sf withContinuation}, which takes
as argument the future for the continuation.

\begin{figure}
\begin{cprog}
class Continuation : public Expression {
public:
	virtual int withContinuation(Continuation *);
	virtual void print() { printf("<future>"); }
	virtual Continuation * isContinuation() { return this; }
};

static Continuation * nothing;	// the null continuation

int Continuation::withContinuation(Continuation * future)
{
	// default is to always work
	return 1;
}
\end{cprog}
\caption{The class {\sf Continuation}}\label{relation}
\end{figure}

Initially there is nothing we want to do in the future.  So the initial
relation simply ignores its future, does nothing and always succeeds.  In fact, 
in our implementation we maintain a
global variable called {\sf nothing} to hold this relation.  You can think
of this variable as maintaining the relation $[$ true $]$.

The simplest relation is the one correspond to the command to print.
When a print relation is created, the value it will eventually print is
saved as part of the relation.
If the argument passed to the print relation is, following any indirection,
a symbolic value than it is printed out, and the future passed to the
relation is invoked.  If the argument was not a symbol, or if the future
calculation was unsuccessful,
then the relation indicates its failure by returning a zero
value.  The code to accomplish this is shown in Figure~\ref{printrel}.

\begin{figure}
\begin{cprog}
class PrintContinuation : public Continuation {
private:
	Expr val;

public:
	PrintContinuation(Expression * x) { val = x; }
	virtual void free() { val = 0; }
	virtual int withContinuation(Continuation *);
};

int PrintContinuation::withContinuation(Continuation * future)
{
	// see if we are a symbol, if so print it out
	Symbol * s = val()->isSymbol();
	if (s) {
		printf("%s\n", s->chars());
		return future->withContinuation(nothing);
		}
	return 0;
}
\end{cprog}
\caption{The print relation}\label{printrel}
\end{figure}

Next let us consider the unification relation.  As with printing, the two
expressions representing the elements to be unified are saved when the
unification operator is encountered during the construction of the future.  
When we invoke this relation the two
arguments are unified, using the algorithm we have previously described.
If this unification is successful the relation 
attempts to evaluate the future continuation.
Only if both of these are successful does
the relation return one.  If either the unification fails or the future
fails then the binding created by the {\sf unify} procedure is undone and
failure is reported.  (Figure~\ref{unifyrel}).

\begin{figure}
\begin{cprog}
class UnifyContinuation : public Continuation {
private:
	Expr left;
	Expr right;
public:
	UnifyContinuation(Expression * a, Expression * b)
		{ left = a; right = b; }
	virtual void free()
		{ left = 0; right = 0; }
	virtual int withContinuation(Continuation *);
};

int UnifyContinuation::withContinuation(Continuation * future)
{
	PrologValue * a = left()->isPrologValue();
	PrologValue * b = right()->isPrologValue();

	// the following shouldn't ever happen, but check anyway
	if ((!a) || (!b)) {
		error("impossible", "missing prolog values in unification");
		return 0;
		}

	// now try unification
	PrologValue * c = 0;
	if (unify(c, a, b) && future->withContinuation(nothing))
		return 1;

	// didn't work, undo assignment and fail
	if (c) 
		c->setUndefined();
	return 0;
}
\end{cprog}
\caption{The unification relation}\label{unifyrel}
\end{figure}

Next consider the {\sf or} relation (Figure~\ref{orrel}).  
This relation takes some number of 
argument relations.  It tries each in turn, followed by the future it has
been provided with.  If any succeeds then it returns a true value,
otherwise if all fail it returns a failure indication.

\begin{figure}
\begin{cprog}
class OrContinuation : public Continuation {
private:
	List relArgs;
public:
	OrContinuation(ListNode * args) { relArgs = args; }
	virtual void free() { relArgs = 0; }
	virtual int withContinuation(Continuation *);
};

int OrContinuation::withContinuation(Continuation * future)
{
	ListNode * args;
	// try each alternative in turn
	for (args = relArgs; ! args->isNil(); args = args->tail()) {
		Continuation * r = args->head()->isContinuation();
		if (! r) {
			error("or argument is non-relation");
			return 0;
			}
		if (r->withContinuation(future)) return 1;
		}
	// nothing worked
	return 0;
}
\end{cprog}
\caption{The {\sf or} relation}\label{orrel}
\end{figure}

It is in the {\sf or} relation that backtracking occurs, although it is
difficult to tell from the code shown here.  Recall that the unification
algorithm undoes the effect of any assignment if the continuation passed to
it cannot be performed.  Thus the future that is passed to the {\sf or} 
relation may be invoked several times before we finally find a sequence of
assignments that works.

The {\sf and} relation is perhaps the most interesting.  To understand this
let us first take the case of only two relations, which we will call {\em
rel1} and {\em rel2}.  Let {\sf f} represent the continuation 
we wish to evaluate if the {\sf and} relation is successful.  What then is
the future we should pass to the first relation?  If the first relation is
successful, we want to evaluate the second relation and then the
continuation.  Thus the future for the first relation is the composition of
the future for the second relation and the original continuation.
This can be written as {\sf rel2(f)}, but we must make it into a
continuation, we we create a new datatype called a {\sf
CompositionContinuation}.  Interestingly, this composition relation ignores
{\em its} continuation, and is merely executed for its side effect.
This is the future we want to pass to the first relation.  
We can generalize this to any number of arguments.  For example the {\em and} 
of three arguments should return the value produced by
{\sf rel1([rel2([rel3([f])])])}, and so on.

The composition step is performed by the datatype {\sf ComposeContinuation},
shown in Figure~\ref{composerel}.  As in our description, when a
composition relation is evaluated it ignores the future it is provided with
and merely returns the first relation provided with the second relation as
its future.  Having defined this, the {\sf and} relation is a simple recursive
invocation.

\begin{figure}
\begin{cprog}
class ComposeContinuation : public Continuation {
private:
	Expr left;
	Expr right;
public:
	ComposeContinuation(Expression * a, Expression * b)
		{ left = a; right = b; }
	virtual void free()
		{ left = 0; right = 0; }
	virtual int withContinuation(Continuation *);
};

int ComposeContinuation::withContinuation(Continuation * future)
{
	Continuation * a = left()->isContinuation();
	Continuation * b = right()->isContinuation();
	if ((! a) || (! b)) {
		error("compose with non relations??");
		return 0;
		}
	return a->withContinuation(b);
}

class AndContinuation : public Continuation {
private:
	List relArgs;
public:
	AndContinuation(ListNode * args)
		{ relArgs = args; }
	virtual void free()
		{ relArgs = 0; }
	virtual int withContinuation(Continuation *);
};

int AndContinuation::withContinuation(Continuation * future)
{
	ListNode * args;
	args = relArgs;
	Continuation * newrel = future;
	for (int i = args->length()-1; i >= 0; i--) 
		newrel = new ComposeContinuation(args->at(i), newrel);

	Expr p = newrel;	// for gc purposes
	int result = newrel->withContinuation(nothing);
	p = 0;
	return result;
}
\end{cprog}
\caption{The and relation}\label{composerel}
\end{figure}

You may have noticed that the class {\sf Continuation} is not a subclass of
class {\sf Function}, and yet we have been discussing continuations as if
they were functions.  This is easily explained.  Recall that evaluating a
relation in our approach is a two-step process.  First the relation is
constructed, and in the second step the future is brought to life.
The functional parts of each of the four relation-building operations
are concerned only with the first part of this task.   These are all
trivial functions, shown in Figure~\ref{relbuild}.

\begin{figure}
\begin{cprog}
class UnifyOperation : public BinaryFunction {
public:
	virtual void applyWithArgs(Expr & target, ListNode * args, 
		Environment *)
		{ target = new UnifyContinuation(args->at(0), args->at(1)); }

};

class PrintOperation : public UnaryFunction {
public:
	virtual void applyWithArgs(Expr & target, ListNode * args, 
		Environment *)
		{ target = new PrintContinuation(args->at(0)); }
};

class AndOperation : public Function {
public:
	virtual void applyWithArgs(Expr & target, ListNode * args, 
		Environment *)
		{ target = new AndContinuation(args); }
};

class OrOperation : public Function {
public:
	virtual void applyWithArgs(Expr & target, ListNode * args, 
		Environment *)
		{ target = new OrContinuation(args); }
};
\end{cprog}
\caption{Building the Relations}\label{relbuild}
\end{figure}

The {\sf query} statement is responsible for the construction and execution
of the continuation corresponding to its argument.  The function
implementing the {\sf query} statement is shown in Figure~\ref{query}.
A new environment is created prior to evaluating the arguments so that
bindings created for new variables do not get entered into the global
environment.  Then the continuation is constructed, simply by evaluating
the argument.  If this process is successful, the continuation is then
executed, and if the continuation is successful the symbol {\sf ok} is
yielded as the result (and thus printed by the read-eval-print loop).
If the continuation is not successful the symbol {\sf not-ok} is generated.

\begin{figure}
\begin{cprog}
void QueryStatement::apply(Expr&target, ListNode*args, Environment*rho)
{
	if (args->length() != 1) {
		target = error("wrong number of args to query");
		return;
		}

	// we make a new environment to isolate any new variables defined
	Env newrho = new Environment(emptyList, emptyList, rho);

	args->at(0)->eval(target, valueOps, newrho);

	Continuation * f = 0;
	if (target())
		f = target()->isContinuation();
	if (! f) {
		target = error("query given non-relation");
		return;
		}
	if (f->withContinuation(nothing))
		target = new Symbol("ok");
	else
		target = new Symbol("not ok");

	newrho = 0;	// force memory management
}
\end{cprog}
\caption{Implementation of the query statement}\label{query}
\end{figure}

The initialization function for the prolog interpreter 
(Figure~\ref{prologinit}) is one of the shortest we have seen.
It is only necessary to create the two commands {\sf define} and {\sf query}, 
and the four relational-building operations.

\begin{figure}
\begin{cprog}
initialize()
{
	// create the reader/parser 
	reader = new PrologReader;

	// make the empty relation
	nothing = new Continuation;

	// make the operators that are legal inside of relations
	Environment * rops = valueOps;
	rops->add(new Symbol("print"), new PrintOperation);
	rops->add(new Symbol(":=:"), new UnifyOperation);
	rops->add(new Symbol("and"), new AndOperation);
	rops->add(new Symbol("or"), new OrOperation);

	// initialize the commands environment
	Environment * cmds = commands;
	cmds->add(new Symbol("define"), new DefineStatement);
	cmds->add(new Symbol("query"), new QueryStatement);
}
\end{cprog}
\caption{Initialization of the Prolog interpreter}\label{prologinit}
\end{figure}

\chapter*{Possible Future Changes}

The following list represents a few of the ideas that occurred to me as I
was developing these interpreters for how things might be done differently.
These are presented in no particular order.  (Nor as any particularly grave
criticism of the Kamin interpreters - I still think the book as a whole is
very good).
\begin{itemize}
\item
The C++ versions of the interpreters have an annoying habit of dumping 
core when an error occurs.  Need to track this down and fix it.
\item
I would remove the while statement from the chapter 2 lisp interpreter.
Students who do not have previous experience with Lisp often have a
difficult time learning to program in a recursive fashion.  For them the
while statement is a crutch, and without it they would be forced to use the
more Lisp-like features of the language.
\item
I would add functionals (called operators in APL) to chapter 3.
Specifically I would make reduction take the function as an argument, and
add inner and outer product.  This would allow an easier transition to
functional programming in the next section.
\item
I might be tempted to add a chapter before chapter 3 on Setl.  This is
another example of a language using large values, and allows a new and
different problem domain to be discussed (namely logic).
\item
It would be nice to add call/cc to the scheme interpreter, but I don't exactly 
see how to do this right now.  
This is not quite as critical now that the Prolog interpreter uses
continuations for its execution.
\item
I would remove the keyword ``rep'' from the CLU syntax, as it is
unnecessary and its elimination simplifies the implementation.
\end{itemize}

\end{document}
